\documentclass{book}

\usepackage[squaren]{SIunits}

\title{Picoquant: Tools for reading and analyzing Picoquant data files}
\author{Thomas Bischof \\ \texttt{tbischof@mit.edu}}
\date{\today}

\newcommand{\cps}{cps}
\newcommand{\braces}[1]{\ensuremath{\left\lbrace #1 \right\rbrace}}
\newcommand{\angles}[1]{\ensuremath{\left\langle #1 \right\rangle}}
\newcommand{\stdin}{\texttt{stdin}}
\newcommand{\stdout}{\texttt{stdout}}

\begin{document}

\maketitle
\tableofcontents
 
\chapter{Introduction}
\section{The layout of this document}
This documents is laid out in roughly three parts:
\begin{enumerate}
\item Overview of terminology and methods
\item Documentation for each program
\item Applications of the software to real problems
\end{enumerate}
In the chapters devoted to the various programs, the documentation is divided further:
\begin{enumerate}
\item Command-line syntax
\item Theoretical overview of the purpose of the program
\item Details of the implementation
\end{enumerate}

\section{A (very) brief overview of the science and technology}
In single-molecule spectroscopy, single-photon detectors are often used to perform time-resolved experiments. Detectors such as silicon-based avalanche photodiodes (APDs) can be used to detect the arrival time a photon with better than 1\nano\second{} resolution, and can detect up to $10^{7}$ photons per second before saturation, abbreviated as 10\mega\cps{} hereafter (counts per second). Hardware capable of resolving these arrival times is of great use for revealing time-dependent structure to the photon stream, such as intensity fluctuations and bunching, so various hardware designs have been developed to permit such measurements.

One line of instruments is the timing hardware produced by Picoquant GmBH, such as the Timeharp, Picoharp, and Hydraharp. These modules are capable of detecting pulse arrivals on multiple input channels with a resolution of as little as 1\pico\second, and operate in a few distinct modes:
\begin{enumerate}
\item Interactive (histogram): one input channel is designated as a sync source, representing a clock-starting signal. On the other channels, pulse arrival times are recorded relative to this clock source, and the times binned into a histogram. 
\item Time-tagged time-resolved (TTTR): 
	\begin{enumerate}
	\item T2: all channels are treated equally, and all pulse arrival times relative to the start of the experiment are recorded. 
	\item T3: this is similar to histogram mode, but instead of binning the arrival times, the sync event number and relative arrival time are both recorded.
	\end{enumerate}
\end{enumerate}

In all cases, the times are discrete and represent some number of cycles of a clock, so all times must be treated as integers representing some number of picoseconds. This has important effects on the definition of time bins for histograms, and where appropriate some time will be devoted to discussing these factors. 

Considering these distinct modes, it is important to spend some time discussing their uses, and how their data should be handled.

\section{Data collection modes}
\subsection{Interactive (histogram)}
A common experiment for studying fluorophores is to measure the time-dependence of their response to an excitation. For example, a pulsed laser with pulse width significantly shorter than the relaxation times of interest can be used to excite a sample, and the resulting fluorescence detected by an APD. Because single-photon detectors are limited to detection of a single photon at once, to rebuild the decay curve it is necessary to average the result over many pulses of the laser, binning the relative arrival times to build a histogram of arrival times representing the time-averaged response of the system under study. If this time-averaged behavior is sufficient, then the interactive histogram mode is used to perform all of this collection and binning on the hardware without any post-processing. 

Consequently, interactive data consists of $N$ arrival time bins with boundaries $(t_{j}, t_{j+1})$ and the number of counts $n_{j}$ associated with that bin, so the data can be represented by the set:
\begin{equation}
\braces{((t_{j}, t_{j+1}), n_{j})}
\end{equation}
The exact choice of where the boundaries lie has some effect on the resulting histogram, but this performed in the hardware and presumably represents $t\in[t_{j}, t_{j+1})$. This choice is not detailed in the manuals for the hardware, but it is of little practical importance; typical detectors have two orders of magnitude more timing jitter than the timing hardware, so the exact definition of the histogram bin has a negligible effect.

\subsection{T2}
In T2 mode, all input channels are connected to photon detectors. At the start of the experiment, an internal clock is reset and started, providing a master timing reference. As a pulse arrives, the machine emits data encoding the channel and time of arrival, so the data follow the form:
\begin{equation}
\braces{(c_{j}, t_{j})}
\end{equation}
for a channel $c_{j}$ and arrival time $t_{j}$ of the $j$th photon. Typically, these data are used to examine time-dependent behavior, such as intensity fluctuations, particularly when the excitation source is continuous-wave. These data are also useful for calculation of correlation functions such as:
\begin{equation}
g^{(n)}(t_{2}, \ldots t_{n}) = \frac
	{\angles{I_{1}(t)\prod_{j=2}^{n}{I_{j}(t+t_{j})}}}
	{I_{1}(t)\prod_{j=2}^{n}{I_{j}(t+t_{j})}}
\end{equation}
The details of this calculation will laid out later, but one use of the $g^{(2)}(t)$ is to determine the number of emitters present in a signal. For example, a single emitter emitting one photon at any time will exhibit so-called antibunching behavior, where $g^{(2)}(t)\rightarrow 0$ for $t\approx 0$, indicating a diminished probability of seeing two successive photon emissions ($g^{(n)}(\braces{t_{j}})=1$ indicates no correlation). Correlations of higher order ($n\ge 3$) have their own uses, but quickly become computationally expensive for reasons which will become clear later.

\subsection{T3}
T3 mode is closest in character to the interactive mode, except that, instead of binning the photon arrival times on the hardware, the arrivals are recorded directly for later examination, as in T2 mode. This gives a data set of the form:
\begin{equation}
\braces{(c_{j}, p_{j}, t_{j})}
\end{equation}
for channel $c_{j}$, sync pulse number $p_{j}$, and relative arrival time $t_{j}$. Indeed, by defining bins and histogramming the $t_{j}$, the result from the interactive mode can be reproduced exactly. However, this mode also allows for studying time dependence and correlation in the data, as with T2 mode. In principle, any data collected in T3 mode can be transformed into T2 mode data if the sync source is regular, as the pulse number will represent some amount of time in the experiment, but there are subtle hardware and numerical issues (discussed later) which limit the practicality of this transformation.

Uses of this mode include the study of the time-dependent fluoresence lifetime of single molecules, which can switch between distinct states under various conditions. Correlation methods can also reveal important behavior, but again this will be discussed later.

\section{General design principles}
Analysis of these timing data can be roughly described as follows:
\begin{enumerate}
\item Produce a stream of photon events
\item Condition the stream by correlation, removal of extraneous information, homogenizing the detection channels, etc.
\item Collect the result to form some histogram of events.
\end{enumerate}
As such, these three phases are handled by distinct programs, which act as filters of a data stream by reading in values and outputting the appropriate new values. These streams are ultimately streams of binary data, but for any program the stream may be defined by some standard data stream (\stdin, \stdout) or by some file containing that data. 

For time-tagged modes, times are represented as integer multiples of 1\pico\second, although only the initial data streamer is actually aware of units; the conditioning and collection routines operate on time as an integer, without regard for its units.

All programs are designed to operate on a data stream until that stream terminates, so any division of a data stream should be handled by a separate program and the result fed into a distinct instance of the handler. 

The software is written in the C programming language, using the C99 standard. Where necessary, the definitions of data types have been defined to be of fixed width, but otherwise the definitions are those minimal for handling reasonable values. For example, in T2 mode time is defined as a signed 64-bit integer, representing $\approx 10^{19}\pico\second$, or $10^{7}\second$, or $106$ days. Pulses are counted as signed 64-bit integers, limiting their total to several thousand years of $10MHz$ pulsed laser excitation. With luck, your experiments will not exceed these values. 

\section{Picoquant}

\subsection{General philosophy and measurement modes}

\subsection{Important hardware and mode details}

\subsection{Implementation in software}

\chapter{Intensity}
\section{Purpose}
For T2 and T3 data, it is often useful to group records into fixed time bins and count the number of records in each bin. This program does just that, reporting the intensity at each channel for all data in a time-ordered stream. 

\section{Command-line syntax}
\begin{verbatim}
Usage: intensity [-v] [-i file_in] [-o file_out] [-c channels]
                 -w bin_width -m mode

           -v, --verbose: Print debug-level information.
           -i, --file-in: Input file. By default, this is
                          STDIN.
          -o, --file-out: Output file. By default, this is
                          STDOUT.
         -w, --bin-width: Width of the bin, in time or pulses
                          depending on the mode.
              -m, --mode: Stream type. This is either t2 or t3,
                          and the style of the output will be
                          different for each.
          -c, --channels: Number of channels in the stream. By
                          default, this is 2.
         -a, --count-all: Rather than counting records in 
                          distinct time bins, count all records 
                          in the stream.
              -h, --help: Print this message.

       This program assumes the input stream is time-ordered.
       
       If counting all events or printing the last one, the time
       written will be the time of the last event seen.
\end{verbatim}

\subsection{Input}
T2 and T3 modes accept data of the form produced by \program{picoquant}, as specified in section~\ref{sec:picoquant_output}.

\subsection{Output}
The output for both T2 and T3 modes is of the form:
\begin{verbatim}
lower time limit, upper time limit, 
  channel 0 intensity, channel 1 intensity, ... \n
\end{verbatim}
For T2 mode, the time bin is defined as a number of picoseconds. For T3 mode, the time bin is a number of pulses.

Passing the flag \texttt{--count-all} will ignore the time bins, and instead count all events on each channel as belonging to one large channel. 

\section{Examples of usage}
Count into 10\milli\second{} bins.
\begin{verbatim}
> picoquant --file-in data.pt2 --number 100000 |  \
  intensity --bin-width 10000000000 --mode t2 --channels 2 
0,10000000000,555,0
10000000000,20000000000,524,0
20000000000,30000000000,564,0
30000000000,40000000000,536,0
40000000000,50000000000,495,0
50000000000,60000000000,530,0
60000000000,70000000000,514,0
70000000000,80000000000,548,0
80000000000,90000000000,558,0
90000000000,100000000000,565,0
100000000000,110000000000,550,0
110000000000,120000000000,490,0
120000000000,130000000000,559,0
130000000000,140000000000,504,0
140000000000,150000000000,547,0
150000000000,160000000000,537,0
160000000000,170000000000,535,0
170000000000,180000000000,545,0
180000000000,186537236212,344,0
\end{verbatim}
Count all events:
\begin{verbatim}
> picoquant --file-in data.pt2 --number 100000 |  \
  intensity --bin-width 10000000000 --mode t2 --channels 2 \
  --count-all
0,186537236212,10000,0
\end{verbatim}

\section{Implementation details}
\label{sec:intensity_implementation}
Each channel can be treated independently, so we can focus on how to handle a single stream of records. 

Given a set of photons $\photon\in\photons$, our goal is to determine the number of $\photon$ whose arrival times are in a time range $\epsilon$:
\begin{equation}
I(\resolution) = \abs{\setbuilder{\photon}{\photon\in\photons;~\Time(\photon)\in\epsilon}}
\end{equation}
In our case, we will only be concerned with time intervals $\resolution$ which are consecutive and collectively span the full integration time:
\begin{align}
\resolution_{j} &= \left[[\resolution_{j}\upminus,\resolution_{j+1}\upplus\right) \\
\integrationtime &= \bigcup\limits_{j=0}^{m}{\resolution_{j}}
\end{align}
As such, any photon can belong to exactly one subset $\photons_{\resolution_{j}}$, and these $\resolution_{j}$ can each be visited exactly once by iterating through times in the experiment. This can be performed efficiently if we impose the condition that the stream of photons be time-ordered, by iterating over photons and bins alternately:
\lstset{language=Python}
\begin{lstlisting}
photon = next(photons)
time_bin = next(time_bins)
intensity = 0

while photon and time_bin:
    if photon in time_bin:
        intensity += 1
        photon = next(photon)
    else:
        yield(time_bin, intensity)
        time_bin = next(time_bins)
        intensity = 0

yield(time_bin, intensity)
\end{lstlisting}
As can be seen, this algorithm scales linearly with the number of time bins and photons, i.e. it scales as O(\abs{\photons}). This algorithm requires time-ordered photons and bins which never lag behind the photon stream, which are guaranteed by appropriate initialization of the photon and bin streams.


\section{Correlate}
\label{sec:correlate}
The details of the mathematics required for understanding this program can be found in chapter~\ref{sec:math_background}. If you have not read that section, the discussion of the implementation here may be difficult to follow, but a brief explanation of the background will be given at each step.
%\subsection{A brief mathematical background}
%\label{sec:correlate_math}
%For several types of experiments, some form of a signal correlation is necessary. These follow the form:
%\begin{equation}
%\label{eq:correlation}
%\gn{n}(\tau_{1}, \ldots \tau_{n-1}) = \frac
%	{\angles{I_{0}(t)\prod_{j=1}^{n-1}{I_{j}(t+\tau_{j})}}}
%	{\angles{I_{0}(t)}\prod_{j=1}^{n-1}{\angles{I_{j}(t+\tau_{j})}}}
%\end{equation}
%where $I_{j}(t)$ are the intensities of some number of signals and the angled brackets indicate an average over time. For some purposes, the autocorrelation of a signal is of interest, where all $I_{j}(t)$ are identical, but in other cases some mixture of signals is of interest, and not necessarily in numerical order. For example, fluoroescence correlation spectroscopy requires the calculation of a second-order autocorrelation of fluorescence intensity, or:
%\begin{equation}
%\label{eq:g2}
%\gn{2}(\tau) = \frac{\angles{I(t)I(t+\tau)}}
%                 {\angles{(I(t)}\angles{(I(t+\tau)}}
%             = \frac{\angles{I(t)I(t+\tau)}}
%                 {\angles{(I(t)}^{2}}
%\end{equation}
%where $I(t)$ includes signal across all detection channels. In most cases, calculation of \gn{2} is sufficient, but in this definition we have implicitly limited our discussion to T2-like signals, with one independent variable per signal. In the case of T3-mode signals, we really have two independent variables: pulse number and arrival time. In principle, we could write something like
%\begin{equation}
%\gn{n}(\rho_{1}, \tau_{1}, \ldots \rho_{n-1}, \tau_{n-1}) = \frac
%	{\angles{I_{0}(p, t)\prod_{j=1}^{n-1}{I_{j}(p+\rho_{j},t)I_{j}(p,t+\tau_{j})}}}
%	{\angles{I_{0}(p, t)}^{2}\prod_{j=1}^{n-1}{\angles{I_{j}(p+\rho_{j},t)}
%	                                       \angles{I_{j}(p,t+\tau_{j})}}}
%\end{equation}
%where the average is now over all pulse and real time. However, because both pulse and real time are integer spaces, we can just map the times onto two homogeneous dimensions, and perform the correlation \gn{2n} instead. This gives a \gn{n} for every choice of pulses, useful for comparisons of long- and short-time correlations, as in multi-excition spectroscopy.

\subsection{Purpose}
This program calculates the full correlation of a signal, including all autocorrelations and cross-correlations of individual channels, for arbitrary numbers of channels and order. The raw correlation events are returned as the output, \textit{not the histogrammed correlation}. The histogramming and normalization are left for other programs, such as \texttt{histogram}.

The speed of the calculation scales favorably with the order of correlation, although of course higher orders of correlation require greater numbers of events to build meaningful results. See section~\ref{sec:correlation_implementation} for details of the algorithm.

\subsection{Command-line syntax}
\section{Correlate}
\label{sec:correlate}
The details of the mathematics required for understanding this program can be found in chapter~\ref{sec:math_background}. If you have not read that section, the discussion of the implementation here may be difficult to follow, but a brief explanation of the background will be given at each step.
%\subsection{A brief mathematical background}
%\label{sec:correlate_math}
%For several types of experiments, some form of a signal correlation is necessary. These follow the form:
%\begin{equation}
%\label{eq:correlation}
%\gn{n}(\tau_{1}, \ldots \tau_{n-1}) = \frac
%	{\angles{I_{0}(t)\prod_{j=1}^{n-1}{I_{j}(t+\tau_{j})}}}
%	{\angles{I_{0}(t)}\prod_{j=1}^{n-1}{\angles{I_{j}(t+\tau_{j})}}}
%\end{equation}
%where $I_{j}(t)$ are the intensities of some number of signals and the angled brackets indicate an average over time. For some purposes, the autocorrelation of a signal is of interest, where all $I_{j}(t)$ are identical, but in other cases some mixture of signals is of interest, and not necessarily in numerical order. For example, fluoroescence correlation spectroscopy requires the calculation of a second-order autocorrelation of fluorescence intensity, or:
%\begin{equation}
%\label{eq:g2}
%\gn{2}(\tau) = \frac{\angles{I(t)I(t+\tau)}}
%                 {\angles{(I(t)}\angles{(I(t+\tau)}}
%             = \frac{\angles{I(t)I(t+\tau)}}
%                 {\angles{(I(t)}^{2}}
%\end{equation}
%where $I(t)$ includes signal across all detection channels. In most cases, calculation of \gn{2} is sufficient, but in this definition we have implicitly limited our discussion to T2-like signals, with one independent variable per signal. In the case of T3-mode signals, we really have two independent variables: pulse number and arrival time. In principle, we could write something like
%\begin{equation}
%\gn{n}(\rho_{1}, \tau_{1}, \ldots \rho_{n-1}, \tau_{n-1}) = \frac
%	{\angles{I_{0}(p, t)\prod_{j=1}^{n-1}{I_{j}(p+\rho_{j},t)I_{j}(p,t+\tau_{j})}}}
%	{\angles{I_{0}(p, t)}^{2}\prod_{j=1}^{n-1}{\angles{I_{j}(p+\rho_{j},t)}
%	                                       \angles{I_{j}(p,t+\tau_{j})}}}
%\end{equation}
%where the average is now over all pulse and real time. However, because both pulse and real time are integer spaces, we can just map the times onto two homogeneous dimensions, and perform the correlation \gn{2n} instead. This gives a \gn{n} for every choice of pulses, useful for comparisons of long- and short-time correlations, as in multi-excition spectroscopy.

\subsection{Purpose}
This program calculates the full correlation of a signal, including all autocorrelations and cross-correlations of individual channels, for arbitrary numbers of channels and order. The raw correlation events are returned as the output, \textit{not the histogrammed correlation}. The histogramming and normalization are left for other programs, such as \texttt{histogram}.

The speed of the calculation scales favorably with the order of correlation, although of course higher orders of correlation require greater numbers of events to build meaningful results. See section~\ref{sec:correlation_implementation} for details of the algorithm.

\subsection{Command-line syntax}
\section{Correlate}
\label{sec:correlate}
The details of the mathematics required for understanding this program can be found in chapter~\ref{sec:math_background}. If you have not read that section, the discussion of the implementation here may be difficult to follow, but a brief explanation of the background will be given at each step.
%\subsection{A brief mathematical background}
%\label{sec:correlate_math}
%For several types of experiments, some form of a signal correlation is necessary. These follow the form:
%\begin{equation}
%\label{eq:correlation}
%\gn{n}(\tau_{1}, \ldots \tau_{n-1}) = \frac
%	{\angles{I_{0}(t)\prod_{j=1}^{n-1}{I_{j}(t+\tau_{j})}}}
%	{\angles{I_{0}(t)}\prod_{j=1}^{n-1}{\angles{I_{j}(t+\tau_{j})}}}
%\end{equation}
%where $I_{j}(t)$ are the intensities of some number of signals and the angled brackets indicate an average over time. For some purposes, the autocorrelation of a signal is of interest, where all $I_{j}(t)$ are identical, but in other cases some mixture of signals is of interest, and not necessarily in numerical order. For example, fluoroescence correlation spectroscopy requires the calculation of a second-order autocorrelation of fluorescence intensity, or:
%\begin{equation}
%\label{eq:g2}
%\gn{2}(\tau) = \frac{\angles{I(t)I(t+\tau)}}
%                 {\angles{(I(t)}\angles{(I(t+\tau)}}
%             = \frac{\angles{I(t)I(t+\tau)}}
%                 {\angles{(I(t)}^{2}}
%\end{equation}
%where $I(t)$ includes signal across all detection channels. In most cases, calculation of \gn{2} is sufficient, but in this definition we have implicitly limited our discussion to T2-like signals, with one independent variable per signal. In the case of T3-mode signals, we really have two independent variables: pulse number and arrival time. In principle, we could write something like
%\begin{equation}
%\gn{n}(\rho_{1}, \tau_{1}, \ldots \rho_{n-1}, \tau_{n-1}) = \frac
%	{\angles{I_{0}(p, t)\prod_{j=1}^{n-1}{I_{j}(p+\rho_{j},t)I_{j}(p,t+\tau_{j})}}}
%	{\angles{I_{0}(p, t)}^{2}\prod_{j=1}^{n-1}{\angles{I_{j}(p+\rho_{j},t)}
%	                                       \angles{I_{j}(p,t+\tau_{j})}}}
%\end{equation}
%where the average is now over all pulse and real time. However, because both pulse and real time are integer spaces, we can just map the times onto two homogeneous dimensions, and perform the correlation \gn{2n} instead. This gives a \gn{n} for every choice of pulses, useful for comparisons of long- and short-time correlations, as in multi-excition spectroscopy.

\subsection{Purpose}
This program calculates the full correlation of a signal, including all autocorrelations and cross-correlations of individual channels, for arbitrary numbers of channels and order. The raw correlation events are returned as the output, \textit{not the histogrammed correlation}. The histogramming and normalization are left for other programs, such as \texttt{histogram}.

The speed of the calculation scales favorably with the order of correlation, although of course higher orders of correlation require greater numbers of events to build meaningful results. See section~\ref{sec:correlation_implementation} for details of the algorithm.

\subsection{Command-line syntax}
\input{programs/correlate.usage}

\subsubsection{Input}
T2 and T3 modes accept data of the form produced by \program{picoquant}, as specified in section~\ref{sec:picoquant_output}.

\subsubsection{Output}
The exact output will depend on the mode and order of correlation, but it always adheres the following form:
\begin{verbatim}
channel 0, channel 1, time 1, ..., channel 2, ... \n
\end{verbatim}
where the times are either times or pulses, and follow the order (pulse, time). Also, the channels here are not necessarily ordered, because the exact order of the signals in the correlation can be of great importance. For example, for a \gn{2} of T2 data:
\begin{verbatim}
channel 0, channel 1, time 1-time 0 \n
\end{verbatim}
for \gn{3} of T2 data:
\begin{verbatim}
channel 0, channel 1, time 1-time 0, 
  channel 2, time 2-time 0 \n
\end{verbatim}
for \gn{3} of T3 data:
\begin{verbatim}
channel 0, channel 1, pulse 1-pulse 0, time 1-time 0,
   channel 2, pulse 2-pulse 0, time 2-time 0 \n
\end{verbatim}

By default, the ordering of the channels is that found in the stream. However, for some applications it is useful to order the channels and record both positive- and negative-time correlations on the same histogram. To account for this, pass the flag \texttt{--channels-ordered} to order the channels and apply the appropriate sign to the time differences.

By default, the program will correlate all entries in a stream. This can easily cause memory problems, so it is recommended that a maximum time distance is specified for T2 data, or a maximum pulse distance for T3 data. A maximum time distance may also be specified for T3 data, but in principle this is unnecessary. Note that all channels are treated equally by these limits. 

Currently, \program{correlate} uses a fixed-length circular buffer to store entries, so if errors report that the buffer is too small try changing the length with \texttt{--queue-size}.

\subsection{Examples of usage}
Finding the correlation events for a \gn{2} of T2 data:
\begin{verbatim}
> picoquant --file-in data.ht2 --number 10000 | \
  correlate --channels 4 --mode t2 \
  --max-time-distance 1000
3,2,932
3,0,558
3,1,508
\end{verbatim}
the same, with channels ordered:
\begin{verbatim}
> picoquant --file-in data.ht2 --number 10000 | \
  correlate --channels 4 --mode t2 \
  --max-time-distance 1000 --channels-ordered
2,3,-932
0,3,-558
1,3,-508
\end{verbatim}
\gn{2} of T3 data:
\begin{verbatim}
> picoquant --file-in data.ht3 --number 100 | \
  correlate --channels 4 --mode t3 \
  --max-pulse-distance 1000
3,3,447,14436
1,1,115,-15636
3,2,369,-28240
1,1,986,66088
0,3,240,21380
\end{verbatim}
\gn{3} of T3 data:
\begin{verbatim}
> picoquant --file-in data.ht3 --number 1000 | \
  correlate --channels 4 --mode t3 --order 2 \
  --max-pulse-distance 1000
3,3,10,-44540,0,913,8648
3,0,155,9044,2,353,-792
2,3,128,47316,3,144,42004
2,2,44,-32120,3,300,1380
2,2,44,-32120,2,359,-4924
2,3,300,1380,2,359,-4924
2,3,256,33500,2,315,27196
1,3,460,72356,2,847,20448
3,2,387,-51908,3,603,-43404
2,1,78,-33380,2,323,-60748
1,3,54,-39056,2,239,-60648
3,2,139,39596,3,364,11248
2,2,27,-82228,3,999,-10108
\end{verbatim}

\subsection{Implementation details}
\label{sec:correlation_implementation}
The math and notation from section~\ref{sec:math_background} will feature heavily in this section, so you should become familiar with the results of that section before proceeding. 

The problem \program{correlate} must address is that of generating efficiently the set of all correlation events, given a set of photons. That is, we must find all events of the form
\begin{equation}
\braces{(\photon_{0},\ldots\photon_{n-1})
        \left|
        \begin{aligned}
        \photon_{0}\in\photons_{c_{0}};\\
        \ldots; \\
        \abs{\braces{\photon_{0},\ldots\photon_{n-1}}}=n;\\
        \Time(\photon_{1})-\Time(\photon_{0})=\tau_{0};\\
        \ldots
        \end{aligned}
        \right.}
\end{equation}
In long form, we must find all tuples of $n$ unique photons which satisfy specific rules for their relative time delays. To do this, we will start with the information we can know or enforce for the incoming stream of photons \photons:
\begin{enumerate}
\item The set \photons{} of all photons is time-ordered ($\Time(\photon_{j})\le\Time(\photon_{k})$ for $j<k$).
\item The correlation will only be meaningful for a time window $\timewindow\subseteq\integrationtime$  much smaller in span than the full integration time.
\end{enumerate}
From this, we see that it should be possible to start from some photon in the stream, find all possible tuples including that photon, then remove it from consideration and move on. That is, for a given photon \photon, there will be some subset of possible times \timewindow{} in which correlation tuples can exist, and once we have exhausted all photons in that window we can ignore the original photon for all future correlations. 

Our task can thus be divided into three main procedures:
\begin{enumerate}
\item Generate each subset $\photons_{\timewindow}$.
\item Generate all unique tuples $\vec{\photon}$ of $n$ photons in a given $\photons_{\timewindow}$.
\item Calculate the relative time delays of the photons in $\vec{\photon}$.
\end{enumerate}
%This latter rule can be expressed as a modification of the conditions in the set:
%\begin{align}
%\photons_{\timewindow}\equiv\braces{\photon\left|\photon\in\photons;~\Time(\photon)\in\timewindow\right.}\\
%\braces{(\photon_{0},\ldots\photon_{n-1})
%        \left|
%        \begin{aligned}
%        \photon_{0}\in\photons_{c_{0}}\cap\photons_{\timewindow};\\
%        \ldots; \\
%        \Time(\photon_{1})-\Time(\photon_{0})=\tau_{0};\\
%        \ldots;
%        \end{aligned}
%        \right.}
%\end{align}
%The former rule simplifies the algorithm we will devise to perform this calculation.

\subsubsection{The set $\photons_{\timewindow}$ can be generated efficiently from a time-ordered \photons}
Given a photon to base correlations from, the particular $\timewindow$ representing the span of possible times can be defined as:
\begin{equation}
\timewindow = \left[ \Time(\photon), \Time(\photon) + \abs{\timewindow} \right)
\end{equation}
Because the photon stream is time-ordered, we can generate the set by the following algorithm:
\begin{lstlisting}
for index, src_photon in enumerate(photons):
    current_photons = list()
    for dst_photon in enumerate(photons[index:]):
        if dst_photon.time - src_photon.time \ 
            < max_distance:
            current_photons.append(dst_photon)
        else:
            break

    correlate(src_photon, current_photons)
\end{lstlisting}
For each photon the stream the algorithm must examine enough photons to reach the end of $\timewindow$, so each event costs an amount proportional to the average density of photons in the stream, or roughyl O($\abs{\timewindow}$). Each photon must be processed, so the whole algorithm costs O($\abs{\photons}\abs{\timewindow}$). In the limit that $\timewindow=\integrationtime$, the algorithm costs O($\abs{\photons}^2$); at worst, for each photon it is necessary to compare with all other photons in the stream.

%Without loss of generality, we define a window window of time $\timewindow$ such that, for some bounds $a,b\in\wholes$, $a<b$:
%\begin{equation}
%\timewindow = [a,b)
%\end{equation}
%If we allow $a\rightarrow0$ and $b\rightarrow\integrationtime$, we recover the full set of times for the experiment, but for practical matters we will focus on this window \timewindow.
%
%Given that \photons{} is time-ordered, we can always draw from it a photon \photon{} with minimal time, so from here we will treat \photons{} as a queue, a structure which holds an arbitrary number of elements in a well-defined order. This queue is time-ordered, although there is some ambiguity in how to order photons arriving on different channels at the same time. Ultimately, the ordering of such photons is not important, but we can define a second order parameter for the channel if needed.
%
%In principle we could enumerate all time windows \timewindow{} with some specified \abs{\timewindow} and find events in that window, but there are far more windows than photons (otherwise, we could have treated the signal as a vector, as in section~\ref{sec:sampling_intensity}). Therefore, it is much more efficient to choose each window as starting at some photon, and draw photons from the queue until the window limit is surpassed. Once we have drawn this sub-queue $\photons_{\timewindow}$ we can turn our attention to generation of photon tuples from that set.
%
%Up to the tuple generation, our algorithm is:
%\begin{enumerate}
%\item[0.] Set $\photons_{\timewindow}=\emptyset$, correlation order $n$.
%\item Draw the next element $\photon$ from $\photons$:
%  \begin{enumerate}
%  \item If $\photon$ exists, add it to the last position in $\photons_{\timewindow}$.
%  \item If no $\photon$ exists (the queue is empty):
%    \begin{enumerate}
%    \item If $\abs{\photons_{\timewindow}}\ge n$, generate tuples from $\photons_{\timewindow}$.
%    \item Halt.
%    \end{enumerate}
%  \end{enumerate}
%\item Call $\photon_{-1}$ the last photon in $\photons_{\timewindow}$ and $\photon_{0}$ the first.
%\item If $\Time(\photon_{-1})-\Time(\photon_{0})\ge\abs{\timewindow}$, go to 1.
%\item If $\abs{\photons_{\timewindow}}\ge n$, generate tuples from $\photons_{\timewindow}$.
%\item Remove the $\photon_{0}$ from $\photons_{\timewindow}$, then go to 1.
%\end{enumerate}
%From this algorithm it is evident that the cost of building the windows scales most directly with the number of photons (we never have to perform more than $\abs{\photons}$ iterations of this algorithm), giving O(\abs{\photons}) for this step. 

To see the implementation of this algorithm, the relevant files are \texttt{correlate.c}, \texttt{correlate\_t2.c}, and \texttt{correlate\_t3.c}. In particular, the functions \texttt{*queue*} are the most relevant, as these handle the population and depopulation of \photons{} and $\photons_{\timewindow}$.

\subsubsection{Generating correlation events (photon tuples) from $\photons_{\timewindow}$}
Given a subset of photons $\photons_{\timewindow}$ and an original tagged photon $\photon_{0}$, we must generate all tuples of $n$ unique photons from this pool. That is, we must generate all unique combinations of $n$ photons from the set $\braces{\photon_{0}}\cup\photons_{\timewindow}$, or equivalently all unique combinations of $n$ choices of  $\abs{\braces{\photon_{0}}\cup\photons_{\timewindow}}$ items. Combination generation algorithms are well-established, and the one used here works by incrementing the trailing digits as needed:
\begin{lstlisting}
combination = range(1, n)
limit = N

def next_combination(combination, limit):
    leading_digit = combination[0]
    
    for index in reversed(range(len(combination))):
        final_index = index
        combination[index] = (combination[index]+1) \
                             % limit
        if combination[index] != 0:
            # No overflow, so we have a valid new value.
            break
    
    if final_index == 0:
        combination[0] = leading_digit + 1
       
    for index in range(final_index, len(combination)):
        combination[index] = combination[index-1] + 1
    
    # If True, the combination is valid. 
    return(combination[-1] < limit) 
\end{lstlisting}
This is a bit obtuse, so we will work out an example. Say we have 4 photons to correlate to order 3. We only want unique photons, so the first combination will be $(0,1,2)$ where we have included the tagged photon for clarity. To get the next combination, we increment the final digit and obtain $(0,1,3)$. This is still valid (all indices are within the limit), so we use it. The next combination is a bit trickier, and we follow these steps to obtain it:
\begin{enumerate}
\item $(1,3+1 \pmod{4}) = (1, 0)$
\item $(1+1, 0) = (2, 0)$
\item We have incremented a value without causing an integer overflow (mod 4), so now we must repopulate the overflowed digits with the smallest possible value.
\item $(2, 2+1) = (2, 3)$
\item $3 < 4$, so this is a valid combination.
\end{enumerate}
This process is repeated up to $(0,2,3)$, at which point $(0,3,4)$ would be generated. As this is not valid, we receive \texttt{False} as output to indicate as such, and the generation stops.

The exact scaling of this algorithm could be worked out by noting the frequency of incrementing $1, 2, \ldots$ digits, but we know it must cost something less than $n$ increments per step. Thus, this generation algorithm will process any given set $\photons_{\timewindow}$ in better than O($n\abs{\photons_{\timewindow}}$) time.

%Given the algorithm to find all subqueues $\photons_{\timewindow}$, we turn our attention to finding the tuples $(\photon_{0},\ldots\photon_{n-1})$ in a given $\photons_{\timewindow}$. Formally, we want to generate the elements of the set
%\begin{equation}
%\braces{(\photon_{0},\ldots\photon_{n-1})\left|\photon_{0},\ldots\in\photons_{\timewindow}\right.}
%\end{equation}
%This is actually a quite familiar problem, if we ignore the photon structure and instead focus on producing the appropriate indices from the queue. Call $N\equiv\abs{\photons_{\timewindow}}$, and note that we are looking to produce all unique permutations of $n$ elements of $\integers_{N}$. These can be ordered to produce:
%\begin{align*}
%&(0, 0,\ldots 0, 0) \\
%&(0, 0, \ldots 0, 1) \\
%&\ldots \\ 
%&(0, 0, \ldots 0, N-1) \\
%&(0, 0, \ldots 1, 0) \\
%&\ldots 
%\end{align*}
%These can be enumerated as the numbers $0,1,\ldots N^{n}-1$ in base $N$, which can be seen by treating each element of the tuple as a coefficient in the base expansion of a number. For example, any number $N\in\wholes$ can be expressed as a sum over powers of 2:
%\begin{equation}
%N = \sum_{j=0}^{n}{c_{j}2^{j}}
%\end{equation}
%for some $n\in\wholes$. While this is true, it is also subject to some redundancy, because we have shown that the positive-time correlation is sufficient for reconstruction all negative times (see section~\ref{sec:correlation_function}), so we actually only need to compute this factor for all time-ordered tuples. Additionally, any tuple containing the same photon twice is not important, because there will never be any structure to a photon correlated with itself. Thus we need only include the tuples for which the indices are ordered and unique, the upper hypertriangle in index space.
%
%Additionally, because the window will move to eliminate the first photon, any correlation not involving this photon should not be handled now. Thus the index tuples of interest are:
%\begin{align*}
%&(0,1,\ldots,n-2, n-1)\\
%&(0,1,\ldots,n-2, n)\\
%&\ldots\\
%&(0,1,\ldots,n-2,N-1)\\
%&(0,1,\ldots,n-1,n)\\
%&\ldots\\
%&(0,N-n,\ldots N-1)
%\end{align*}
%
%To generate these combinations of indices, apply the following algorithm:
%\begin{enumerate}
%\item[0.] Define $x\leftarrow(0,1,\ldots n-1)$, and index the $j$th element of $x$ as $x_{j}$. 
%\item Yield $x$.
%\item Increment as many digits of $x$ as necessary:
%  \begin{enumerate}
%  \item Set $j\leftarrow n-1$.  
%  \item If $j\le 1$, halt this loop.
%  \item Increment $x_{j}$. 
%  \item If $x_{j}\ge N$, set $x_{j}\leftarrow 0$, decrement $j$, and go to 1(b) (overflow of this digit).
%  \item Otherwise, set $j\leftarrow 0$, decrement $j$, and go to 1(b) (no overflow, stop incrementing).
%  \end{enumerate}
%\item Refill the overflowed digits of $x$:
% \begin{enumerate}
% \item If $x_{1}=0$, we have incremented up to the first index. Set $x_{1}\leftarrow x_{1}'+1$.
% \item Set $j\leftarrow 1$.
% \item If $j\ge n$, halt this loop.
% \item If $x_{j}=0$, set $x_{j}\leftarrow x_{j-1}+1$ (increment the digit based on the previous non-overflowed digit).
% \item Increment $j$, and go to 3(c).
% \end{enumerate}
%\item If $x_{n-1}\ge N$, halt. This is not a valid tuple of indices, which means we have exhausted all of the valid tuples.
%\item Otherwise, go to 1.
%\end{enumerate}
%This algorithm is not perfect, as we must loop over all elements of the tuple in the refilling step. This is ultimately a minor cost and could be accounted for in the implementation, but for \program{correlate} the algorithm was implemented as described.
%
%In long form, we start at the right end of the tuple, incrementing values while moving left until we find a value which does not overflow past $N$. Once we find this value, we stop incrementing, then form the next possible tuple from the leading non-zero index, or the previous value of the leading digit if we have overflowed that as well. If the tuple can be refilled without overflowing, it is valid and we yield it. Otherwise, we have reached the end and are finished.
%
%From the algorithm, we see that an increment costs, at worst, O($n$) loops. In reality the average is somewhat lower than this, and could be calculated exactly with some effort. Additionally, this algorithm will produce some number of tuples of somewhat less than order O($nN$). This behavior is sensible: a larger window ($N=\abs{\photons_{\timewindow}}$) gives more valid tuples of photons, and the higher order $n$ gives more iterations over that window.
%
To see the implementation of this algorithm in \program{correlate}, examine the functions \texttt{*offsets*} in \texttt{combinations.c}.

\subsubsection{Production of the correlation event, given a photon tuple}
Given a tuple of $n$ photons, all correlation events corresponding to that tuple can be achieved by determining the displacements between all permutations of those photons. For example, two photons $\photon_{0}$ and $\photon_{1}$ will contribute both to $\gn{2}_{(0,1)}$ and $\gn{2}_{(1,0)}$, so we must consider the $\tau$ produced by treating each ordering of the two photons. Generation of permutations is an equivalent problem to generation of integers in some base, such that no two digits are identical. In the code, this is achieved by incrementing through all $n$-digit numbers in base $n$, and storing the ones with no identical digits in an array for later usage. This requires the generation of $n^{n}$ values and the storage of $n!$ values, but this should be a small number in most circumstances. 

To see how the permutations are generated, see \texttt{combinations.c}. To see how these are used, see \texttt{correlate\_t*.c}, particularly the \texttt{correlate\_t*\_block} procedures.
%There are two distinct modes enabled in \program{correlate} for producing the correlation events: with and without ordering of channels. We will deal with the time-ordered, channel-unordered case first. 
%
%Given a tuple of photons $(\photon_{0},\ldots\photon_{n-1})$ known to satisfy the conditions for including in calculating \gn{n}, we must produce the time differences $(\tau_{1}, \ldots\tau_{n-1})$. This can be done with the following algorithm:
%\begin{enumerate}
%\item[0.] Set $j\leftarrow 1$.
%\item If $j\ge n$, halt and yield $(\tau_{1},\ldots\tau_{n-1})$.
%\item Set $\tau_{j}\leftarrow \Time(\photon_{j}) - \Time(\photon_{0})$.
%\item Increment $j$ and go to 1.
%\end{enumerate}
%This process scales as O($n$), although the constant factor is quite small relative to the other steps in the overall algorithm. Correlation for T3 mode involves a second term for the pulse difference, but it otherwise identical.
%
%To perform this same step for channel-ordered (time-unordered) photons, we can pre-populate a list of all possible channel orders and combinations and the order of the indices to choose, instead of incrementing from start to finish. The details of this will be described in section~\ref{sec:histogram}, but the result is that the cost of this operation is only upfront, and there is essentially no additional cost once the list of combinations exists.

%\subsubsection{Definition of the correlation as the order of a set}
%As mentioned in section~\ref{sec:correlate_math}, T3 and T2 data are closely related and can be treated similarly. To understand how this is possible, it is worth spending some time considering what information is required to compute the correlation.
%
%In equation~\ref{eq:g2}, the denominator is an average intensity of a signal. This can be computed simply by determining the duration of the signal and the number of counts over that interval, and can be handled by using the result of \intensity{} \texttt{--count-all}. Thus the real problem is the computation of the numerator, which is somewhat imposing at first glance:
%\begin{equation}
%\angles{I_{0}(t)\prod_{j=1}^{n-1}{I_{j}(t+\tau_{j})}}
%\end{equation}
%To simplify matters, we will start with the autocorrelation of a signal channel:
%\begin{equation}
%\angles{I(t)I(t+\tau)}
%\end{equation}
%In this situation, the signal can be thought of as being composed of some sum of delta functions with peak centers at the times photons arrived. Thus the contribution of any given pair of photons to the correlation at time $\tau$ is nonzero only if the difference of their arrival times is $\tau$. Thus, for a given value of $\tau$ and photon arrival times \braces{t}:
%\begin{equation}
%\gn{2}(\tau) \propto \left| \braces{(t_{j}, t_{k}) | t_{j},t_{k}\in\braces{t}; t_{j}-t_{k}=\tau} \right|
%\end{equation}
%This result can be extended to higher orders by enforcing the restriction that a tuple of time differences must be satisfied:
%\begin{equation}
%\begin{split}
%\gn{n}(\tau_{1}, &\ldots \tau_{n-1}) \propto \\
%   &\left| \braces{
%       (t_{0}, \ldots t_{n-1})
%       \left|\begin{split}
%       t_{0},\ldots t_{n-1}\in\braces{t}; \\
%       (t_{1}, \ldots t_{n-1}) - (t_{0}, \ldots t_{0}) = (\tau_{1}, \ldots \tau_{n-1})
%       \end{split}\right.
%%       \begin{gather*}
%%        \\
%%       
%%       \end{gather*}
%   } \right|
%\end{split}
%\end{equation}
%
%\subsubsection{Mapping T3 data onto T2-like data}
%Now that we have a general expression for calculation of \gn{n}, it is worthwhile to make an aside describing how to map T3 data onto T2-like data for correlation. Consider the basic form of a tuple of $n$ T3 records:
%\begin{equation}
%\left((c_{0}, p_{0}, t_{0}), \ldots (c_{n-1}, p_{n-1}, t_{n-1})\right)
%\end{equation}
%which can be mapped isomorphically onto:
%\begin{equation}
%\left((c_{0}, p_{0}), (c_{0}, t_{0}), \ldots (c_{n-1}, p_{n-1}), (c_{n-1}, t_{n-1})\right)
%\end{equation}
%which looks like a tuple for T2 records:
%\begin{equation}
%\left((c_{0}, t_{0}), \ldots (c_{n-1}, t_{n-1})\right)
%\end{equation}
%Ultimately, a T3 tuple of length $n$ can be mapped onto a corresponding tuple of length $2n$, such that any correlation \gn{n} of T3 data can be treated exactly as a corresponding correlation \gn{2n} of T2 data.
%
%\subsubsection{Correlation of the time-ordered stream}
%As with \intensity, \program{correlate} expects a time-ordered stream (see section~\ref{sec:intensity_implementation}). 
%
%% For a correlation of order $n$, every unique combination of $n$ elements of the stream can contribute to \gn{n}, so the full correlation of stream of lengh $N$  costs O(N$^{n}$) to compute. This is reduced for a fixed window width to approximately O(Nw$^{n-1}$) for a window of width w, for reasons that will become clear shortly.


\subsubsection{Input}
T2 and T3 modes accept data of the form produced by \program{picoquant}, as specified in section~\ref{sec:picoquant_output}.

\subsubsection{Output}
The exact output will depend on the mode and order of correlation, but it always adheres the following form:
\begin{verbatim}
channel 0, channel 1, time 1, ..., channel 2, ... \n
\end{verbatim}
where the times are either times or pulses, and follow the order (pulse, time). Also, the channels here are not necessarily ordered, because the exact order of the signals in the correlation can be of great importance. For example, for a \gn{2} of T2 data:
\begin{verbatim}
channel 0, channel 1, time 1-time 0 \n
\end{verbatim}
for \gn{3} of T2 data:
\begin{verbatim}
channel 0, channel 1, time 1-time 0, 
  channel 2, time 2-time 0 \n
\end{verbatim}
for \gn{3} of T3 data:
\begin{verbatim}
channel 0, channel 1, pulse 1-pulse 0, time 1-time 0,
   channel 2, pulse 2-pulse 0, time 2-time 0 \n
\end{verbatim}

By default, the ordering of the channels is that found in the stream. However, for some applications it is useful to order the channels and record both positive- and negative-time correlations on the same histogram. To account for this, pass the flag \texttt{--channels-ordered} to order the channels and apply the appropriate sign to the time differences.

By default, the program will correlate all entries in a stream. This can easily cause memory problems, so it is recommended that a maximum time distance is specified for T2 data, or a maximum pulse distance for T3 data. A maximum time distance may also be specified for T3 data, but in principle this is unnecessary. Note that all channels are treated equally by these limits. 

Currently, \program{correlate} uses a fixed-length circular buffer to store entries, so if errors report that the buffer is too small try changing the length with \texttt{--queue-size}.

\subsection{Examples of usage}
Finding the correlation events for a \gn{2} of T2 data:
\begin{verbatim}
> picoquant --file-in data.ht2 --number 10000 | \
  correlate --channels 4 --mode t2 \
  --max-time-distance 1000
3,2,932
3,0,558
3,1,508
\end{verbatim}
the same, with channels ordered:
\begin{verbatim}
> picoquant --file-in data.ht2 --number 10000 | \
  correlate --channels 4 --mode t2 \
  --max-time-distance 1000 --channels-ordered
2,3,-932
0,3,-558
1,3,-508
\end{verbatim}
\gn{2} of T3 data:
\begin{verbatim}
> picoquant --file-in data.ht3 --number 100 | \
  correlate --channels 4 --mode t3 \
  --max-pulse-distance 1000
3,3,447,14436
1,1,115,-15636
3,2,369,-28240
1,1,986,66088
0,3,240,21380
\end{verbatim}
\gn{3} of T3 data:
\begin{verbatim}
> picoquant --file-in data.ht3 --number 1000 | \
  correlate --channels 4 --mode t3 --order 2 \
  --max-pulse-distance 1000
3,3,10,-44540,0,913,8648
3,0,155,9044,2,353,-792
2,3,128,47316,3,144,42004
2,2,44,-32120,3,300,1380
2,2,44,-32120,2,359,-4924
2,3,300,1380,2,359,-4924
2,3,256,33500,2,315,27196
1,3,460,72356,2,847,20448
3,2,387,-51908,3,603,-43404
2,1,78,-33380,2,323,-60748
1,3,54,-39056,2,239,-60648
3,2,139,39596,3,364,11248
2,2,27,-82228,3,999,-10108
\end{verbatim}

\subsection{Implementation details}
\label{sec:correlation_implementation}
The math and notation from section~\ref{sec:math_background} will feature heavily in this section, so you should become familiar with the results of that section before proceeding. 

The problem \program{correlate} must address is that of generating efficiently the set of all correlation events, given a set of photons. That is, we must find all events of the form
\begin{equation}
\braces{(\photon_{0},\ldots\photon_{n-1})
        \left|
        \begin{aligned}
        \photon_{0}\in\photons_{c_{0}};\\
        \ldots; \\
        \abs{\braces{\photon_{0},\ldots\photon_{n-1}}}=n;\\
        \Time(\photon_{1})-\Time(\photon_{0})=\tau_{0};\\
        \ldots
        \end{aligned}
        \right.}
\end{equation}
In long form, we must find all tuples of $n$ unique photons which satisfy specific rules for their relative time delays. To do this, we will start with the information we can know or enforce for the incoming stream of photons \photons:
\begin{enumerate}
\item The set \photons{} of all photons is time-ordered ($\Time(\photon_{j})\le\Time(\photon_{k})$ for $j<k$).
\item The correlation will only be meaningful for a time window $\timewindow\subseteq\integrationtime$  much smaller in span than the full integration time.
\end{enumerate}
From this, we see that it should be possible to start from some photon in the stream, find all possible tuples including that photon, then remove it from consideration and move on. That is, for a given photon \photon, there will be some subset of possible times \timewindow{} in which correlation tuples can exist, and once we have exhausted all photons in that window we can ignore the original photon for all future correlations. 

Our task can thus be divided into three main procedures:
\begin{enumerate}
\item Generate each subset $\photons_{\timewindow}$.
\item Generate all unique tuples $\vec{\photon}$ of $n$ photons in a given $\photons_{\timewindow}$.
\item Calculate the relative time delays of the photons in $\vec{\photon}$.
\end{enumerate}
%This latter rule can be expressed as a modification of the conditions in the set:
%\begin{align}
%\photons_{\timewindow}\equiv\braces{\photon\left|\photon\in\photons;~\Time(\photon)\in\timewindow\right.}\\
%\braces{(\photon_{0},\ldots\photon_{n-1})
%        \left|
%        \begin{aligned}
%        \photon_{0}\in\photons_{c_{0}}\cap\photons_{\timewindow};\\
%        \ldots; \\
%        \Time(\photon_{1})-\Time(\photon_{0})=\tau_{0};\\
%        \ldots;
%        \end{aligned}
%        \right.}
%\end{align}
%The former rule simplifies the algorithm we will devise to perform this calculation.

\subsubsection{The set $\photons_{\timewindow}$ can be generated efficiently from a time-ordered \photons}
Given a photon to base correlations from, the particular $\timewindow$ representing the span of possible times can be defined as:
\begin{equation}
\timewindow = \left[ \Time(\photon), \Time(\photon) + \abs{\timewindow} \right)
\end{equation}
Because the photon stream is time-ordered, we can generate the set by the following algorithm:
\begin{lstlisting}
for index, src_photon in enumerate(photons):
    current_photons = list()
    for dst_photon in enumerate(photons[index:]):
        if dst_photon.time - src_photon.time \ 
            < max_distance:
            current_photons.append(dst_photon)
        else:
            break

    correlate(src_photon, current_photons)
\end{lstlisting}
For each photon the stream the algorithm must examine enough photons to reach the end of $\timewindow$, so each event costs an amount proportional to the average density of photons in the stream, or roughyl O($\abs{\timewindow}$). Each photon must be processed, so the whole algorithm costs O($\abs{\photons}\abs{\timewindow}$). In the limit that $\timewindow=\integrationtime$, the algorithm costs O($\abs{\photons}^2$); at worst, for each photon it is necessary to compare with all other photons in the stream.

%Without loss of generality, we define a window window of time $\timewindow$ such that, for some bounds $a,b\in\wholes$, $a<b$:
%\begin{equation}
%\timewindow = [a,b)
%\end{equation}
%If we allow $a\rightarrow0$ and $b\rightarrow\integrationtime$, we recover the full set of times for the experiment, but for practical matters we will focus on this window \timewindow.
%
%Given that \photons{} is time-ordered, we can always draw from it a photon \photon{} with minimal time, so from here we will treat \photons{} as a queue, a structure which holds an arbitrary number of elements in a well-defined order. This queue is time-ordered, although there is some ambiguity in how to order photons arriving on different channels at the same time. Ultimately, the ordering of such photons is not important, but we can define a second order parameter for the channel if needed.
%
%In principle we could enumerate all time windows \timewindow{} with some specified \abs{\timewindow} and find events in that window, but there are far more windows than photons (otherwise, we could have treated the signal as a vector, as in section~\ref{sec:sampling_intensity}). Therefore, it is much more efficient to choose each window as starting at some photon, and draw photons from the queue until the window limit is surpassed. Once we have drawn this sub-queue $\photons_{\timewindow}$ we can turn our attention to generation of photon tuples from that set.
%
%Up to the tuple generation, our algorithm is:
%\begin{enumerate}
%\item[0.] Set $\photons_{\timewindow}=\emptyset$, correlation order $n$.
%\item Draw the next element $\photon$ from $\photons$:
%  \begin{enumerate}
%  \item If $\photon$ exists, add it to the last position in $\photons_{\timewindow}$.
%  \item If no $\photon$ exists (the queue is empty):
%    \begin{enumerate}
%    \item If $\abs{\photons_{\timewindow}}\ge n$, generate tuples from $\photons_{\timewindow}$.
%    \item Halt.
%    \end{enumerate}
%  \end{enumerate}
%\item Call $\photon_{-1}$ the last photon in $\photons_{\timewindow}$ and $\photon_{0}$ the first.
%\item If $\Time(\photon_{-1})-\Time(\photon_{0})\ge\abs{\timewindow}$, go to 1.
%\item If $\abs{\photons_{\timewindow}}\ge n$, generate tuples from $\photons_{\timewindow}$.
%\item Remove the $\photon_{0}$ from $\photons_{\timewindow}$, then go to 1.
%\end{enumerate}
%From this algorithm it is evident that the cost of building the windows scales most directly with the number of photons (we never have to perform more than $\abs{\photons}$ iterations of this algorithm), giving O(\abs{\photons}) for this step. 

To see the implementation of this algorithm, the relevant files are \texttt{correlate.c}, \texttt{correlate\_t2.c}, and \texttt{correlate\_t3.c}. In particular, the functions \texttt{*queue*} are the most relevant, as these handle the population and depopulation of \photons{} and $\photons_{\timewindow}$.

\subsubsection{Generating correlation events (photon tuples) from $\photons_{\timewindow}$}
Given a subset of photons $\photons_{\timewindow}$ and an original tagged photon $\photon_{0}$, we must generate all tuples of $n$ unique photons from this pool. That is, we must generate all unique combinations of $n$ photons from the set $\braces{\photon_{0}}\cup\photons_{\timewindow}$, or equivalently all unique combinations of $n$ choices of  $\abs{\braces{\photon_{0}}\cup\photons_{\timewindow}}$ items. Combination generation algorithms are well-established, and the one used here works by incrementing the trailing digits as needed:
\begin{lstlisting}
combination = range(1, n)
limit = N

def next_combination(combination, limit):
    leading_digit = combination[0]
    
    for index in reversed(range(len(combination))):
        final_index = index
        combination[index] = (combination[index]+1) \
                             % limit
        if combination[index] != 0:
            # No overflow, so we have a valid new value.
            break
    
    if final_index == 0:
        combination[0] = leading_digit + 1
       
    for index in range(final_index, len(combination)):
        combination[index] = combination[index-1] + 1
    
    # If True, the combination is valid. 
    return(combination[-1] < limit) 
\end{lstlisting}
This is a bit obtuse, so we will work out an example. Say we have 4 photons to correlate to order 3. We only want unique photons, so the first combination will be $(0,1,2)$ where we have included the tagged photon for clarity. To get the next combination, we increment the final digit and obtain $(0,1,3)$. This is still valid (all indices are within the limit), so we use it. The next combination is a bit trickier, and we follow these steps to obtain it:
\begin{enumerate}
\item $(1,3+1 \pmod{4}) = (1, 0)$
\item $(1+1, 0) = (2, 0)$
\item We have incremented a value without causing an integer overflow (mod 4), so now we must repopulate the overflowed digits with the smallest possible value.
\item $(2, 2+1) = (2, 3)$
\item $3 < 4$, so this is a valid combination.
\end{enumerate}
This process is repeated up to $(0,2,3)$, at which point $(0,3,4)$ would be generated. As this is not valid, we receive \texttt{False} as output to indicate as such, and the generation stops.

The exact scaling of this algorithm could be worked out by noting the frequency of incrementing $1, 2, \ldots$ digits, but we know it must cost something less than $n$ increments per step. Thus, this generation algorithm will process any given set $\photons_{\timewindow}$ in better than O($n\abs{\photons_{\timewindow}}$) time.

%Given the algorithm to find all subqueues $\photons_{\timewindow}$, we turn our attention to finding the tuples $(\photon_{0},\ldots\photon_{n-1})$ in a given $\photons_{\timewindow}$. Formally, we want to generate the elements of the set
%\begin{equation}
%\braces{(\photon_{0},\ldots\photon_{n-1})\left|\photon_{0},\ldots\in\photons_{\timewindow}\right.}
%\end{equation}
%This is actually a quite familiar problem, if we ignore the photon structure and instead focus on producing the appropriate indices from the queue. Call $N\equiv\abs{\photons_{\timewindow}}$, and note that we are looking to produce all unique permutations of $n$ elements of $\integers_{N}$. These can be ordered to produce:
%\begin{align*}
%&(0, 0,\ldots 0, 0) \\
%&(0, 0, \ldots 0, 1) \\
%&\ldots \\ 
%&(0, 0, \ldots 0, N-1) \\
%&(0, 0, \ldots 1, 0) \\
%&\ldots 
%\end{align*}
%These can be enumerated as the numbers $0,1,\ldots N^{n}-1$ in base $N$, which can be seen by treating each element of the tuple as a coefficient in the base expansion of a number. For example, any number $N\in\wholes$ can be expressed as a sum over powers of 2:
%\begin{equation}
%N = \sum_{j=0}^{n}{c_{j}2^{j}}
%\end{equation}
%for some $n\in\wholes$. While this is true, it is also subject to some redundancy, because we have shown that the positive-time correlation is sufficient for reconstruction all negative times (see section~\ref{sec:correlation_function}), so we actually only need to compute this factor for all time-ordered tuples. Additionally, any tuple containing the same photon twice is not important, because there will never be any structure to a photon correlated with itself. Thus we need only include the tuples for which the indices are ordered and unique, the upper hypertriangle in index space.
%
%Additionally, because the window will move to eliminate the first photon, any correlation not involving this photon should not be handled now. Thus the index tuples of interest are:
%\begin{align*}
%&(0,1,\ldots,n-2, n-1)\\
%&(0,1,\ldots,n-2, n)\\
%&\ldots\\
%&(0,1,\ldots,n-2,N-1)\\
%&(0,1,\ldots,n-1,n)\\
%&\ldots\\
%&(0,N-n,\ldots N-1)
%\end{align*}
%
%To generate these combinations of indices, apply the following algorithm:
%\begin{enumerate}
%\item[0.] Define $x\leftarrow(0,1,\ldots n-1)$, and index the $j$th element of $x$ as $x_{j}$. 
%\item Yield $x$.
%\item Increment as many digits of $x$ as necessary:
%  \begin{enumerate}
%  \item Set $j\leftarrow n-1$.  
%  \item If $j\le 1$, halt this loop.
%  \item Increment $x_{j}$. 
%  \item If $x_{j}\ge N$, set $x_{j}\leftarrow 0$, decrement $j$, and go to 1(b) (overflow of this digit).
%  \item Otherwise, set $j\leftarrow 0$, decrement $j$, and go to 1(b) (no overflow, stop incrementing).
%  \end{enumerate}
%\item Refill the overflowed digits of $x$:
% \begin{enumerate}
% \item If $x_{1}=0$, we have incremented up to the first index. Set $x_{1}\leftarrow x_{1}'+1$.
% \item Set $j\leftarrow 1$.
% \item If $j\ge n$, halt this loop.
% \item If $x_{j}=0$, set $x_{j}\leftarrow x_{j-1}+1$ (increment the digit based on the previous non-overflowed digit).
% \item Increment $j$, and go to 3(c).
% \end{enumerate}
%\item If $x_{n-1}\ge N$, halt. This is not a valid tuple of indices, which means we have exhausted all of the valid tuples.
%\item Otherwise, go to 1.
%\end{enumerate}
%This algorithm is not perfect, as we must loop over all elements of the tuple in the refilling step. This is ultimately a minor cost and could be accounted for in the implementation, but for \program{correlate} the algorithm was implemented as described.
%
%In long form, we start at the right end of the tuple, incrementing values while moving left until we find a value which does not overflow past $N$. Once we find this value, we stop incrementing, then form the next possible tuple from the leading non-zero index, or the previous value of the leading digit if we have overflowed that as well. If the tuple can be refilled without overflowing, it is valid and we yield it. Otherwise, we have reached the end and are finished.
%
%From the algorithm, we see that an increment costs, at worst, O($n$) loops. In reality the average is somewhat lower than this, and could be calculated exactly with some effort. Additionally, this algorithm will produce some number of tuples of somewhat less than order O($nN$). This behavior is sensible: a larger window ($N=\abs{\photons_{\timewindow}}$) gives more valid tuples of photons, and the higher order $n$ gives more iterations over that window.
%
To see the implementation of this algorithm in \program{correlate}, examine the functions \texttt{*offsets*} in \texttt{combinations.c}.

\subsubsection{Production of the correlation event, given a photon tuple}
Given a tuple of $n$ photons, all correlation events corresponding to that tuple can be achieved by determining the displacements between all permutations of those photons. For example, two photons $\photon_{0}$ and $\photon_{1}$ will contribute both to $\gn{2}_{(0,1)}$ and $\gn{2}_{(1,0)}$, so we must consider the $\tau$ produced by treating each ordering of the two photons. Generation of permutations is an equivalent problem to generation of integers in some base, such that no two digits are identical. In the code, this is achieved by incrementing through all $n$-digit numbers in base $n$, and storing the ones with no identical digits in an array for later usage. This requires the generation of $n^{n}$ values and the storage of $n!$ values, but this should be a small number in most circumstances. 

To see how the permutations are generated, see \texttt{combinations.c}. To see how these are used, see \texttt{correlate\_t*.c}, particularly the \texttt{correlate\_t*\_block} procedures.
%There are two distinct modes enabled in \program{correlate} for producing the correlation events: with and without ordering of channels. We will deal with the time-ordered, channel-unordered case first. 
%
%Given a tuple of photons $(\photon_{0},\ldots\photon_{n-1})$ known to satisfy the conditions for including in calculating \gn{n}, we must produce the time differences $(\tau_{1}, \ldots\tau_{n-1})$. This can be done with the following algorithm:
%\begin{enumerate}
%\item[0.] Set $j\leftarrow 1$.
%\item If $j\ge n$, halt and yield $(\tau_{1},\ldots\tau_{n-1})$.
%\item Set $\tau_{j}\leftarrow \Time(\photon_{j}) - \Time(\photon_{0})$.
%\item Increment $j$ and go to 1.
%\end{enumerate}
%This process scales as O($n$), although the constant factor is quite small relative to the other steps in the overall algorithm. Correlation for T3 mode involves a second term for the pulse difference, but it otherwise identical.
%
%To perform this same step for channel-ordered (time-unordered) photons, we can pre-populate a list of all possible channel orders and combinations and the order of the indices to choose, instead of incrementing from start to finish. The details of this will be described in section~\ref{sec:histogram}, but the result is that the cost of this operation is only upfront, and there is essentially no additional cost once the list of combinations exists.

%\subsubsection{Definition of the correlation as the order of a set}
%As mentioned in section~\ref{sec:correlate_math}, T3 and T2 data are closely related and can be treated similarly. To understand how this is possible, it is worth spending some time considering what information is required to compute the correlation.
%
%In equation~\ref{eq:g2}, the denominator is an average intensity of a signal. This can be computed simply by determining the duration of the signal and the number of counts over that interval, and can be handled by using the result of \intensity{} \texttt{--count-all}. Thus the real problem is the computation of the numerator, which is somewhat imposing at first glance:
%\begin{equation}
%\angles{I_{0}(t)\prod_{j=1}^{n-1}{I_{j}(t+\tau_{j})}}
%\end{equation}
%To simplify matters, we will start with the autocorrelation of a signal channel:
%\begin{equation}
%\angles{I(t)I(t+\tau)}
%\end{equation}
%In this situation, the signal can be thought of as being composed of some sum of delta functions with peak centers at the times photons arrived. Thus the contribution of any given pair of photons to the correlation at time $\tau$ is nonzero only if the difference of their arrival times is $\tau$. Thus, for a given value of $\tau$ and photon arrival times \braces{t}:
%\begin{equation}
%\gn{2}(\tau) \propto \left| \braces{(t_{j}, t_{k}) | t_{j},t_{k}\in\braces{t}; t_{j}-t_{k}=\tau} \right|
%\end{equation}
%This result can be extended to higher orders by enforcing the restriction that a tuple of time differences must be satisfied:
%\begin{equation}
%\begin{split}
%\gn{n}(\tau_{1}, &\ldots \tau_{n-1}) \propto \\
%   &\left| \braces{
%       (t_{0}, \ldots t_{n-1})
%       \left|\begin{split}
%       t_{0},\ldots t_{n-1}\in\braces{t}; \\
%       (t_{1}, \ldots t_{n-1}) - (t_{0}, \ldots t_{0}) = (\tau_{1}, \ldots \tau_{n-1})
%       \end{split}\right.
%%       \begin{gather*}
%%        \\
%%       
%%       \end{gather*}
%   } \right|
%\end{split}
%\end{equation}
%
%\subsubsection{Mapping T3 data onto T2-like data}
%Now that we have a general expression for calculation of \gn{n}, it is worthwhile to make an aside describing how to map T3 data onto T2-like data for correlation. Consider the basic form of a tuple of $n$ T3 records:
%\begin{equation}
%\left((c_{0}, p_{0}, t_{0}), \ldots (c_{n-1}, p_{n-1}, t_{n-1})\right)
%\end{equation}
%which can be mapped isomorphically onto:
%\begin{equation}
%\left((c_{0}, p_{0}), (c_{0}, t_{0}), \ldots (c_{n-1}, p_{n-1}), (c_{n-1}, t_{n-1})\right)
%\end{equation}
%which looks like a tuple for T2 records:
%\begin{equation}
%\left((c_{0}, t_{0}), \ldots (c_{n-1}, t_{n-1})\right)
%\end{equation}
%Ultimately, a T3 tuple of length $n$ can be mapped onto a corresponding tuple of length $2n$, such that any correlation \gn{n} of T3 data can be treated exactly as a corresponding correlation \gn{2n} of T2 data.
%
%\subsubsection{Correlation of the time-ordered stream}
%As with \intensity, \program{correlate} expects a time-ordered stream (see section~\ref{sec:intensity_implementation}). 
%
%% For a correlation of order $n$, every unique combination of $n$ elements of the stream can contribute to \gn{n}, so the full correlation of stream of lengh $N$  costs O(N$^{n}$) to compute. This is reduced for a fixed window width to approximately O(Nw$^{n-1}$) for a window of width w, for reasons that will become clear shortly.


\subsubsection{Input}
T2 and T3 modes accept data of the form produced by \program{picoquant}, as specified in section~\ref{sec:picoquant_output}.

\subsubsection{Output}
The exact output will depend on the mode and order of correlation, but it always adheres the following form:
\begin{verbatim}
channel 0, channel 1, time 1, ..., channel 2, ... \n
\end{verbatim}
where the times are either times or pulses, and follow the order (pulse, time). Also, the channels here are not necessarily ordered, because the exact order of the signals in the correlation can be of great importance. For example, for a \gn{2} of T2 data:
\begin{verbatim}
channel 0, channel 1, time 1-time 0 \n
\end{verbatim}
for \gn{3} of T2 data:
\begin{verbatim}
channel 0, channel 1, time 1-time 0, 
  channel 2, time 2-time 0 \n
\end{verbatim}
for \gn{3} of T3 data:
\begin{verbatim}
channel 0, channel 1, pulse 1-pulse 0, time 1-time 0,
   channel 2, pulse 2-pulse 0, time 2-time 0 \n
\end{verbatim}

By default, the ordering of the channels is that found in the stream. However, for some applications it is useful to order the channels and record both positive- and negative-time correlations on the same histogram. To account for this, pass the flag \texttt{--channels-ordered} to order the channels and apply the appropriate sign to the time differences.

By default, the program will correlate all entries in a stream. This can easily cause memory problems, so it is recommended that a maximum time distance is specified for T2 data, or a maximum pulse distance for T3 data. A maximum time distance may also be specified for T3 data, but in principle this is unnecessary. Note that all channels are treated equally by these limits. 

Currently, \program{correlate} uses a fixed-length circular buffer to store entries, so if errors report that the buffer is too small try changing the length with \texttt{--queue-size}.

\subsection{Examples of usage}
Finding the correlation events for a \gn{2} of T2 data:
\begin{verbatim}
> picoquant --file-in data.ht2 --number 10000 | \
  correlate --channels 4 --mode t2 \
  --max-time-distance 1000
3,2,932
3,0,558
3,1,508
\end{verbatim}
the same, with channels ordered:
\begin{verbatim}
> picoquant --file-in data.ht2 --number 10000 | \
  correlate --channels 4 --mode t2 \
  --max-time-distance 1000 --channels-ordered
2,3,-932
0,3,-558
1,3,-508
\end{verbatim}
\gn{2} of T3 data:
\begin{verbatim}
> picoquant --file-in data.ht3 --number 100 | \
  correlate --channels 4 --mode t3 \
  --max-pulse-distance 1000
3,3,447,14436
1,1,115,-15636
3,2,369,-28240
1,1,986,66088
0,3,240,21380
\end{verbatim}
\gn{3} of T3 data:
\begin{verbatim}
> picoquant --file-in data.ht3 --number 1000 | \
  correlate --channels 4 --mode t3 --order 2 \
  --max-pulse-distance 1000
3,3,10,-44540,0,913,8648
3,0,155,9044,2,353,-792
2,3,128,47316,3,144,42004
2,2,44,-32120,3,300,1380
2,2,44,-32120,2,359,-4924
2,3,300,1380,2,359,-4924
2,3,256,33500,2,315,27196
1,3,460,72356,2,847,20448
3,2,387,-51908,3,603,-43404
2,1,78,-33380,2,323,-60748
1,3,54,-39056,2,239,-60648
3,2,139,39596,3,364,11248
2,2,27,-82228,3,999,-10108
\end{verbatim}

\subsection{Implementation details}
\label{sec:correlation_implementation}
The math and notation from section~\ref{sec:math_background} will feature heavily in this section, so you should become familiar with the results of that section before proceeding. 

The problem \program{correlate} must address is that of generating efficiently the set of all correlation events, given a set of photons. That is, we must find all events of the form
\begin{equation}
\braces{(\photon_{0},\ldots\photon_{n-1})
        \left|
        \begin{aligned}
        \photon_{0}\in\photons_{c_{0}};\\
        \ldots; \\
        \abs{\braces{\photon_{0},\ldots\photon_{n-1}}}=n;\\
        \Time(\photon_{1})-\Time(\photon_{0})=\tau_{0};\\
        \ldots
        \end{aligned}
        \right.}
\end{equation}
In long form, we must find all tuples of $n$ unique photons which satisfy specific rules for their relative time delays. To do this, we will start with the information we can know or enforce for the incoming stream of photons \photons:
\begin{enumerate}
\item The set \photons{} of all photons is time-ordered ($\Time(\photon_{j})\le\Time(\photon_{k})$ for $j<k$).
\item The correlation will only be meaningful for a time window $\timewindow\subseteq\integrationtime$  much smaller in span than the full integration time.
\end{enumerate}
From this, we see that it should be possible to start from some photon in the stream, find all possible tuples including that photon, then remove it from consideration and move on. That is, for a given photon \photon, there will be some subset of possible times \timewindow{} in which correlation tuples can exist, and once we have exhausted all photons in that window we can ignore the original photon for all future correlations. 

Our task can thus be divided into three main procedures:
\begin{enumerate}
\item Generate each subset $\photons_{\timewindow}$.
\item Generate all unique tuples $\vec{\photon}$ of $n$ photons in a given $\photons_{\timewindow}$.
\item Calculate the relative time delays of the photons in $\vec{\photon}$.
\end{enumerate}
%This latter rule can be expressed as a modification of the conditions in the set:
%\begin{align}
%\photons_{\timewindow}\equiv\braces{\photon\left|\photon\in\photons;~\Time(\photon)\in\timewindow\right.}\\
%\braces{(\photon_{0},\ldots\photon_{n-1})
%        \left|
%        \begin{aligned}
%        \photon_{0}\in\photons_{c_{0}}\cap\photons_{\timewindow};\\
%        \ldots; \\
%        \Time(\photon_{1})-\Time(\photon_{0})=\tau_{0};\\
%        \ldots;
%        \end{aligned}
%        \right.}
%\end{align}
%The former rule simplifies the algorithm we will devise to perform this calculation.

\subsubsection{The set $\photons_{\timewindow}$ can be generated efficiently from a time-ordered \photons}
Given a photon to base correlations from, the particular $\timewindow$ representing the span of possible times can be defined as:
\begin{equation}
\timewindow = \left[ \Time(\photon), \Time(\photon) + \abs{\timewindow} \right)
\end{equation}
Because the photon stream is time-ordered, we can generate the set by the following algorithm:
\begin{lstlisting}
for index, src_photon in enumerate(photons):
    current_photons = list()
    for dst_photon in enumerate(photons[index:]):
        if dst_photon.time - src_photon.time \ 
            < max_distance:
            current_photons.append(dst_photon)
        else:
            break

    correlate(src_photon, current_photons)
\end{lstlisting}
For each photon the stream the algorithm must examine enough photons to reach the end of $\timewindow$, so each event costs an amount proportional to the average density of photons in the stream, or roughyl O($\abs{\timewindow}$). Each photon must be processed, so the whole algorithm costs O($\abs{\photons}\abs{\timewindow}$). In the limit that $\timewindow=\integrationtime$, the algorithm costs O($\abs{\photons}^2$); at worst, for each photon it is necessary to compare with all other photons in the stream.

%Without loss of generality, we define a window window of time $\timewindow$ such that, for some bounds $a,b\in\wholes$, $a<b$:
%\begin{equation}
%\timewindow = [a,b)
%\end{equation}
%If we allow $a\rightarrow0$ and $b\rightarrow\integrationtime$, we recover the full set of times for the experiment, but for practical matters we will focus on this window \timewindow.
%
%Given that \photons{} is time-ordered, we can always draw from it a photon \photon{} with minimal time, so from here we will treat \photons{} as a queue, a structure which holds an arbitrary number of elements in a well-defined order. This queue is time-ordered, although there is some ambiguity in how to order photons arriving on different channels at the same time. Ultimately, the ordering of such photons is not important, but we can define a second order parameter for the channel if needed.
%
%In principle we could enumerate all time windows \timewindow{} with some specified \abs{\timewindow} and find events in that window, but there are far more windows than photons (otherwise, we could have treated the signal as a vector, as in section~\ref{sec:sampling_intensity}). Therefore, it is much more efficient to choose each window as starting at some photon, and draw photons from the queue until the window limit is surpassed. Once we have drawn this sub-queue $\photons_{\timewindow}$ we can turn our attention to generation of photon tuples from that set.
%
%Up to the tuple generation, our algorithm is:
%\begin{enumerate}
%\item[0.] Set $\photons_{\timewindow}=\emptyset$, correlation order $n$.
%\item Draw the next element $\photon$ from $\photons$:
%  \begin{enumerate}
%  \item If $\photon$ exists, add it to the last position in $\photons_{\timewindow}$.
%  \item If no $\photon$ exists (the queue is empty):
%    \begin{enumerate}
%    \item If $\abs{\photons_{\timewindow}}\ge n$, generate tuples from $\photons_{\timewindow}$.
%    \item Halt.
%    \end{enumerate}
%  \end{enumerate}
%\item Call $\photon_{-1}$ the last photon in $\photons_{\timewindow}$ and $\photon_{0}$ the first.
%\item If $\Time(\photon_{-1})-\Time(\photon_{0})\ge\abs{\timewindow}$, go to 1.
%\item If $\abs{\photons_{\timewindow}}\ge n$, generate tuples from $\photons_{\timewindow}$.
%\item Remove the $\photon_{0}$ from $\photons_{\timewindow}$, then go to 1.
%\end{enumerate}
%From this algorithm it is evident that the cost of building the windows scales most directly with the number of photons (we never have to perform more than $\abs{\photons}$ iterations of this algorithm), giving O(\abs{\photons}) for this step. 

To see the implementation of this algorithm, the relevant files are \texttt{correlate.c}, \texttt{correlate\_t2.c}, and \texttt{correlate\_t3.c}. In particular, the functions \texttt{*queue*} are the most relevant, as these handle the population and depopulation of \photons{} and $\photons_{\timewindow}$.

\subsubsection{Generating correlation events (photon tuples) from $\photons_{\timewindow}$}
Given a subset of photons $\photons_{\timewindow}$ and an original tagged photon $\photon_{0}$, we must generate all tuples of $n$ unique photons from this pool. That is, we must generate all unique combinations of $n$ photons from the set $\braces{\photon_{0}}\cup\photons_{\timewindow}$, or equivalently all unique combinations of $n$ choices of  $\abs{\braces{\photon_{0}}\cup\photons_{\timewindow}}$ items. Combination generation algorithms are well-established, and the one used here works by incrementing the trailing digits as needed:
\begin{lstlisting}
combination = range(1, n)
limit = N

def next_combination(combination, limit):
    leading_digit = combination[0]
    
    for index in reversed(range(len(combination))):
        final_index = index
        combination[index] = (combination[index]+1) \
                             % limit
        if combination[index] != 0:
            # No overflow, so we have a valid new value.
            break
    
    if final_index == 0:
        combination[0] = leading_digit + 1
       
    for index in range(final_index, len(combination)):
        combination[index] = combination[index-1] + 1
    
    # If True, the combination is valid. 
    return(combination[-1] < limit) 
\end{lstlisting}
This is a bit obtuse, so we will work out an example. Say we have 4 photons to correlate to order 3. We only want unique photons, so the first combination will be $(0,1,2)$ where we have included the tagged photon for clarity. To get the next combination, we increment the final digit and obtain $(0,1,3)$. This is still valid (all indices are within the limit), so we use it. The next combination is a bit trickier, and we follow these steps to obtain it:
\begin{enumerate}
\item $(1,3+1 \pmod{4}) = (1, 0)$
\item $(1+1, 0) = (2, 0)$
\item We have incremented a value without causing an integer overflow (mod 4), so now we must repopulate the overflowed digits with the smallest possible value.
\item $(2, 2+1) = (2, 3)$
\item $3 < 4$, so this is a valid combination.
\end{enumerate}
This process is repeated up to $(0,2,3)$, at which point $(0,3,4)$ would be generated. As this is not valid, we receive \texttt{False} as output to indicate as such, and the generation stops.

The exact scaling of this algorithm could be worked out by noting the frequency of incrementing $1, 2, \ldots$ digits, but we know it must cost something less than $n$ increments per step. Thus, this generation algorithm will process any given set $\photons_{\timewindow}$ in better than O($n\abs{\photons_{\timewindow}}$) time.

%Given the algorithm to find all subqueues $\photons_{\timewindow}$, we turn our attention to finding the tuples $(\photon_{0},\ldots\photon_{n-1})$ in a given $\photons_{\timewindow}$. Formally, we want to generate the elements of the set
%\begin{equation}
%\braces{(\photon_{0},\ldots\photon_{n-1})\left|\photon_{0},\ldots\in\photons_{\timewindow}\right.}
%\end{equation}
%This is actually a quite familiar problem, if we ignore the photon structure and instead focus on producing the appropriate indices from the queue. Call $N\equiv\abs{\photons_{\timewindow}}$, and note that we are looking to produce all unique permutations of $n$ elements of $\integers_{N}$. These can be ordered to produce:
%\begin{align*}
%&(0, 0,\ldots 0, 0) \\
%&(0, 0, \ldots 0, 1) \\
%&\ldots \\ 
%&(0, 0, \ldots 0, N-1) \\
%&(0, 0, \ldots 1, 0) \\
%&\ldots 
%\end{align*}
%These can be enumerated as the numbers $0,1,\ldots N^{n}-1$ in base $N$, which can be seen by treating each element of the tuple as a coefficient in the base expansion of a number. For example, any number $N\in\wholes$ can be expressed as a sum over powers of 2:
%\begin{equation}
%N = \sum_{j=0}^{n}{c_{j}2^{j}}
%\end{equation}
%for some $n\in\wholes$. While this is true, it is also subject to some redundancy, because we have shown that the positive-time correlation is sufficient for reconstruction all negative times (see section~\ref{sec:correlation_function}), so we actually only need to compute this factor for all time-ordered tuples. Additionally, any tuple containing the same photon twice is not important, because there will never be any structure to a photon correlated with itself. Thus we need only include the tuples for which the indices are ordered and unique, the upper hypertriangle in index space.
%
%Additionally, because the window will move to eliminate the first photon, any correlation not involving this photon should not be handled now. Thus the index tuples of interest are:
%\begin{align*}
%&(0,1,\ldots,n-2, n-1)\\
%&(0,1,\ldots,n-2, n)\\
%&\ldots\\
%&(0,1,\ldots,n-2,N-1)\\
%&(0,1,\ldots,n-1,n)\\
%&\ldots\\
%&(0,N-n,\ldots N-1)
%\end{align*}
%
%To generate these combinations of indices, apply the following algorithm:
%\begin{enumerate}
%\item[0.] Define $x\leftarrow(0,1,\ldots n-1)$, and index the $j$th element of $x$ as $x_{j}$. 
%\item Yield $x$.
%\item Increment as many digits of $x$ as necessary:
%  \begin{enumerate}
%  \item Set $j\leftarrow n-1$.  
%  \item If $j\le 1$, halt this loop.
%  \item Increment $x_{j}$. 
%  \item If $x_{j}\ge N$, set $x_{j}\leftarrow 0$, decrement $j$, and go to 1(b) (overflow of this digit).
%  \item Otherwise, set $j\leftarrow 0$, decrement $j$, and go to 1(b) (no overflow, stop incrementing).
%  \end{enumerate}
%\item Refill the overflowed digits of $x$:
% \begin{enumerate}
% \item If $x_{1}=0$, we have incremented up to the first index. Set $x_{1}\leftarrow x_{1}'+1$.
% \item Set $j\leftarrow 1$.
% \item If $j\ge n$, halt this loop.
% \item If $x_{j}=0$, set $x_{j}\leftarrow x_{j-1}+1$ (increment the digit based on the previous non-overflowed digit).
% \item Increment $j$, and go to 3(c).
% \end{enumerate}
%\item If $x_{n-1}\ge N$, halt. This is not a valid tuple of indices, which means we have exhausted all of the valid tuples.
%\item Otherwise, go to 1.
%\end{enumerate}
%This algorithm is not perfect, as we must loop over all elements of the tuple in the refilling step. This is ultimately a minor cost and could be accounted for in the implementation, but for \program{correlate} the algorithm was implemented as described.
%
%In long form, we start at the right end of the tuple, incrementing values while moving left until we find a value which does not overflow past $N$. Once we find this value, we stop incrementing, then form the next possible tuple from the leading non-zero index, or the previous value of the leading digit if we have overflowed that as well. If the tuple can be refilled without overflowing, it is valid and we yield it. Otherwise, we have reached the end and are finished.
%
%From the algorithm, we see that an increment costs, at worst, O($n$) loops. In reality the average is somewhat lower than this, and could be calculated exactly with some effort. Additionally, this algorithm will produce some number of tuples of somewhat less than order O($nN$). This behavior is sensible: a larger window ($N=\abs{\photons_{\timewindow}}$) gives more valid tuples of photons, and the higher order $n$ gives more iterations over that window.
%
To see the implementation of this algorithm in \program{correlate}, examine the functions \texttt{*offsets*} in \texttt{combinations.c}.

\subsubsection{Production of the correlation event, given a photon tuple}
Given a tuple of $n$ photons, all correlation events corresponding to that tuple can be achieved by determining the displacements between all permutations of those photons. For example, two photons $\photon_{0}$ and $\photon_{1}$ will contribute both to $\gn{2}_{(0,1)}$ and $\gn{2}_{(1,0)}$, so we must consider the $\tau$ produced by treating each ordering of the two photons. Generation of permutations is an equivalent problem to generation of integers in some base, such that no two digits are identical. In the code, this is achieved by incrementing through all $n$-digit numbers in base $n$, and storing the ones with no identical digits in an array for later usage. This requires the generation of $n^{n}$ values and the storage of $n!$ values, but this should be a small number in most circumstances. 

To see how the permutations are generated, see \texttt{combinations.c}. To see how these are used, see \texttt{correlate\_t*.c}, particularly the \texttt{correlate\_t*\_block} procedures.
%There are two distinct modes enabled in \program{correlate} for producing the correlation events: with and without ordering of channels. We will deal with the time-ordered, channel-unordered case first. 
%
%Given a tuple of photons $(\photon_{0},\ldots\photon_{n-1})$ known to satisfy the conditions for including in calculating \gn{n}, we must produce the time differences $(\tau_{1}, \ldots\tau_{n-1})$. This can be done with the following algorithm:
%\begin{enumerate}
%\item[0.] Set $j\leftarrow 1$.
%\item If $j\ge n$, halt and yield $(\tau_{1},\ldots\tau_{n-1})$.
%\item Set $\tau_{j}\leftarrow \Time(\photon_{j}) - \Time(\photon_{0})$.
%\item Increment $j$ and go to 1.
%\end{enumerate}
%This process scales as O($n$), although the constant factor is quite small relative to the other steps in the overall algorithm. Correlation for T3 mode involves a second term for the pulse difference, but it otherwise identical.
%
%To perform this same step for channel-ordered (time-unordered) photons, we can pre-populate a list of all possible channel orders and combinations and the order of the indices to choose, instead of incrementing from start to finish. The details of this will be described in section~\ref{sec:histogram}, but the result is that the cost of this operation is only upfront, and there is essentially no additional cost once the list of combinations exists.

%\subsubsection{Definition of the correlation as the order of a set}
%As mentioned in section~\ref{sec:correlate_math}, T3 and T2 data are closely related and can be treated similarly. To understand how this is possible, it is worth spending some time considering what information is required to compute the correlation.
%
%In equation~\ref{eq:g2}, the denominator is an average intensity of a signal. This can be computed simply by determining the duration of the signal and the number of counts over that interval, and can be handled by using the result of \intensity{} \texttt{--count-all}. Thus the real problem is the computation of the numerator, which is somewhat imposing at first glance:
%\begin{equation}
%\angles{I_{0}(t)\prod_{j=1}^{n-1}{I_{j}(t+\tau_{j})}}
%\end{equation}
%To simplify matters, we will start with the autocorrelation of a signal channel:
%\begin{equation}
%\angles{I(t)I(t+\tau)}
%\end{equation}
%In this situation, the signal can be thought of as being composed of some sum of delta functions with peak centers at the times photons arrived. Thus the contribution of any given pair of photons to the correlation at time $\tau$ is nonzero only if the difference of their arrival times is $\tau$. Thus, for a given value of $\tau$ and photon arrival times \braces{t}:
%\begin{equation}
%\gn{2}(\tau) \propto \left| \braces{(t_{j}, t_{k}) | t_{j},t_{k}\in\braces{t}; t_{j}-t_{k}=\tau} \right|
%\end{equation}
%This result can be extended to higher orders by enforcing the restriction that a tuple of time differences must be satisfied:
%\begin{equation}
%\begin{split}
%\gn{n}(\tau_{1}, &\ldots \tau_{n-1}) \propto \\
%   &\left| \braces{
%       (t_{0}, \ldots t_{n-1})
%       \left|\begin{split}
%       t_{0},\ldots t_{n-1}\in\braces{t}; \\
%       (t_{1}, \ldots t_{n-1}) - (t_{0}, \ldots t_{0}) = (\tau_{1}, \ldots \tau_{n-1})
%       \end{split}\right.
%%       \begin{gather*}
%%        \\
%%       
%%       \end{gather*}
%   } \right|
%\end{split}
%\end{equation}
%
%\subsubsection{Mapping T3 data onto T2-like data}
%Now that we have a general expression for calculation of \gn{n}, it is worthwhile to make an aside describing how to map T3 data onto T2-like data for correlation. Consider the basic form of a tuple of $n$ T3 records:
%\begin{equation}
%\left((c_{0}, p_{0}, t_{0}), \ldots (c_{n-1}, p_{n-1}, t_{n-1})\right)
%\end{equation}
%which can be mapped isomorphically onto:
%\begin{equation}
%\left((c_{0}, p_{0}), (c_{0}, t_{0}), \ldots (c_{n-1}, p_{n-1}), (c_{n-1}, t_{n-1})\right)
%\end{equation}
%which looks like a tuple for T2 records:
%\begin{equation}
%\left((c_{0}, t_{0}), \ldots (c_{n-1}, t_{n-1})\right)
%\end{equation}
%Ultimately, a T3 tuple of length $n$ can be mapped onto a corresponding tuple of length $2n$, such that any correlation \gn{n} of T3 data can be treated exactly as a corresponding correlation \gn{2n} of T2 data.
%
%\subsubsection{Correlation of the time-ordered stream}
%As with \intensity, \program{correlate} expects a time-ordered stream (see section~\ref{sec:intensity_implementation}). 
%
%% For a correlation of order $n$, every unique combination of $n$ elements of the stream can contribute to \gn{n}, so the full correlation of stream of lengh $N$  costs O(N$^{n}$) to compute. This is reduced for a fixed window width to approximately O(Nw$^{n-1}$) for a window of width w, for reasons that will become clear shortly.

\section{Histogram}
\label{sec:histogram}

\subsection{Purpose}
This program performs a multi-dimensional histogram of photon correlation events. These are defined by the program \program{correlate}, though it is also possible to histogram T3 photons directly. The output is a set of histogram bins and the number of events which fall into each. This output is \textit{not} normalized by bin width or any other factor, and represents the raw number of counts falling into each bin.

For T3 photons, \program{histogram} can be used to build a histogram of events as would be done in interactive mode. To do this, set the mode to T3, and order to 1.

For \gn{n>2}, all time dimensions are defined identically, and all pulse dimensions are defined identically. If distinct time dimensions are required, the modification can be achieved at the stage where bins are defined, without any modification later in the code.

\subsection{Command-line syntax}
\section{Histogram}
\label{sec:histogram}

\subsection{Purpose}
This program performs a multi-dimensional histogram of photon correlation events. These are defined by the program \program{correlate}, though it is also possible to histogram T3 photons directly. The output is a set of histogram bins and the number of events which fall into each. This output is \textit{not} normalized by bin width or any other factor, and represents the raw number of counts falling into each bin.

For T3 photons, \program{histogram} can be used to build a histogram of events as would be done in interactive mode. To do this, set the mode to T3, and order to 1.

For \gn{n>2}, all time dimensions are defined identically, and all pulse dimensions are defined identically. If distinct time dimensions are required, the modification can be achieved at the stage where bins are defined, without any modification later in the code.

\subsection{Command-line syntax}
\section{Histogram}
\label{sec:histogram}

\subsection{Purpose}
This program performs a multi-dimensional histogram of photon correlation events. These are defined by the program \program{correlate}, though it is also possible to histogram T3 photons directly. The output is a set of histogram bins and the number of events which fall into each. This output is \textit{not} normalized by bin width or any other factor, and represents the raw number of counts falling into each bin.

For T3 photons, \program{histogram} can be used to build a histogram of events as would be done in interactive mode. To do this, set the mode to T3, and order to 1.

For \gn{n>2}, all time dimensions are defined identically, and all pulse dimensions are defined identically. If distinct time dimensions are required, the modification can be achieved at the stage where bins are defined, without any modification later in the code.

\subsection{Command-line syntax}
\input{programs/histogram.usage}

\subsubsection{Input}
\paragraph{\gn{1} of T3 data}
The expected input is a stream of T3 photons.

\paragraph{\gn{n\ge 2}}
The expected input is the output of \program{correlate}.

\paragraph{General options}
The main options specified for \program{histogram} define the axes for pulse and time. These are treated equivalently, so we will focus on the time axes.

For \texttt{--time}, the parameters specify the lower and upper bounds of the time axis, as well as the number of bins $n$ to create along that axis. For a linear spacing, the bin width $\Delta t$ is $(t_{\max}-t_{\min})/n$, such that the bins are defined by the ranges
\begin{equation}
\begin{aligned}
&[\time_{\min},\time_{\min}+\Delta\time),\\
&[\time_{\min}+\Delta\time,\time_{\min}+2\Delta\time),\\
&\ldots\\
&[\time_{\min}+(n-1)\Delta\time,\time_{\min}+n\Delta\time)
\end{aligned}
\end{equation}
This linear spacing is the default behavior, but the flag \texttt{--time-scale} can produce two other scales: log, and log-zero. The log scale creates bins with fixed width over the span of $[\log(\time_{\min}),\log(\time_{\max}))$, as if often desired for measurements requiring long and short time correlations. The log-zero scale has identical behavior, except that any zero-time correlations ($\timedelay=0$) are placed into the first bin. Note that the log scale cannot handle zero-time correlations, and neither log not log-zero can handle negative-time correlations. These values will be dropped from the histogram, with an error message indicating this has happened.

\subsubsection{Output}
After the input stream terminates, \program{histogram} outputs the bin definitions and the number of counts associated with that bin. Generically, this format is:
\begin{verbatim}
channel 0, channel 1, bin (1,1) lower, bin (1,1) upper, ...,
    channel 2, ... , 
    counts \n
\end{verbatim}
where the channels are integers, bin edges are floats, and the counts are integers. For every bin in the histogram, one line will be output. For T2 mode, the bin definition has only one dimension (time), so the format is:
\begin{verbatim}
channel 0, channel 1, time 1 lower, time 1 upper, 
           channel 2, time 2 lower, ...,
           counts \n
\end{verbatim}
T3 data have an additional dimension (pulse):
\begin{verbatim}
channel 0, channel 1, pulse 1 lower, pulse 1 upper,
                      time 1 lower, time 1 uppper,
           channel 2, ...
           counts \n
\end{verbatim}
See section~\ref{sec:histogram_examples} for specific examples of output in these formats.


\subsection{Examples of usage}
\label{sec:histogram_examples}
\subsubsection{Time-averaged photoluminescence lifetime from T3 data}
In T3 mode, a correlation order \gn{1} is code for interactive-like behavior. Formally this is a correlation of the laser pulse and the system response, but this language is not often used.
\begin{verbatim}
> picoquant --file-in data.pt3 |  \
  histogram --mode t3 --order 1 --channels 2 \
            --time 0,10,500000
0,0.00,50000.00,0
0,50000.00,100000.00,0
...
1,0.00,50000.00,11267838
1,50000.00,100000.00,14947845
1,100000.00,150000.00,1512803
1,150000.00,200000.00,1152498
1,200000.00,250000.00,1037717
1,250000.00,300000.00,973572
1,300000.00,350000.00,932802
1,350000.00,400000.00,899615
1,400000.00,450000.00,12278
1,450000.00,500000.00,0
\end{verbatim}
Note that \texttt{histogram} will operate on channel 0 as well, even though channel 1 is the only channel with any signal. This costs extra memory and some computational overhead at startup, but ultimately the cost is insignificant compared to the cost of processing the data stream.

\subsubsection{\gn{2} from T2 data}
This data represents an electronic sync source (channel 4) and the detection of the laser itself.
\begin{verbatim}
> picoquant --file-in data.ht2 | \
  correlate --mode t2 --order 2 \
            --channels 5 \
            --max-time-distance 1000 | \
  histogram --mode t2 --order 2 \
            --channels 5 \
            --time 0,10,1000
0,0,0.00,100.00,0
...
0,4,0.00,100.00,43
0,4,100.00,200.00,24
0,4,200.00,300.00,38
0,4,300.00,400.00,43
0,4,400.00,500.00,44
...
4,4,900.00,1000.00,0
\end{verbatim}

\subsubsection{\gn{2} from T3 data}
\begin{verbatim}
> picoquant --file-in data.pt3 | \
  correlate --mode t3 --channels 2 \
            --order 2 \
            --max-pulse-distance 3 \
  histogram --mode t3 --channels 2 \
            --order 2 --pulse 0,3,3 \
            --time -500000,2,500000
0,0,0.00,1.00,-500000.00,0.00,0
...
1,1,0.00,1.00,-500000.00,0.00,0
1,1,0.00,1.00,0.00,500000.00,104640
1,1,1.00,2.00,-500000.00,0.00,123289
1,1,1.00,2.00,0.00,500000.00,124676
1,1,2.00,3.00,-500000.00,0.00,231962
1,1,2.00,3.00,0.00,500000.00,231694
\end{verbatim}
As a rough benchmark, on a computer with a dual-core 3\giga\hertz{} processor running 32-bit Linux, this command required 25\second{} of wall time, for \texttt{data.ht3} containing 32.7 million photon records (18.1\kilo\cps) for a total of 0.8 million correlation events.

\subsubsection{\gn{3} from T2 data}
This T2 data is the same as the laser data from before:
\begin{verbatim}
> picoquant --file-in data.ht2 | \
  correlate --mode t2 --order 2 \
            --channels 5 \
            --max-time-distance 1000000 | \
  histogram --mode t2 --order 2 \
            --channels 5 \
            --time 0,1,1000000
0,0,0.00,1000000.00,0,0.00,1000000.00,31
0,0,0.00,1000000.00,1,0.00,1000000.00,55
0,0,0.00,1000000.00,2,0.00,1000000.00,57
...
4,0,0.00,1000000.00,2,0.00,1000000.00,2710
4,0,0.00,1000000.00,3,0.00,1000000.00,2235
4,0,0.00,1000000.00,4,0.00,1000000.00,0
...
\end{verbatim}
Note how much larger the time window must be to catch significant numbers of higher-order events.

\subsection{Implementation details}
The problem of populating and returning histograms for all possible correlations can be broken into a few distinct steps:
\begin{enumerate}
\item Construct a histogram for every permutation of channels.
\item For each correlation event:
  \begin{enumerate}
  \item Identify the histogram corresponding to the permutation of channels.
  \item Identify the bin associated with the parameters of the correlation.
  \item Increment that bin in that histogram.
  \end{enumerate}
\item For every bin of every histogram, print the bin definition and associated counts
\end{enumerate}
As such, each of these will be discussed separately. The discussion will focus on T2 data, but T3 data are handled identically for twice the order of equivalent T2 data.

\subsubsection{The cross-correlations can be enumerated as base-$\abs{\channels}$ numbers}
Each histogram corresponds to a single cross-correlation, as identified by the tuple $\vec{\channel}\in\channels^{n}$, for $\channels$ the set of all channels and correlation order $n$. As such, if we enumerate the channels as whole numbers ($0, 1, \ldots$), it is evident that each element of the tuple can be treated as the coefficient of an $n$-digit number in base $\abs{\channels}$. Mapping these tuples onto the set of whole numbers can thus be achieved by the following formula:
\begin{equation}
\Index(\vec{\channel}) = \sum_{j=0}^{n-1}{\channel_{j}\abs{\channels}^{n-1-j}}
\end{equation}
where the elements $\channel$ refer implicitly to the index of that channel in our enumeration.
If $\abs{\channels}=2$, this is identical to the expression of an $n$-digit binary number.

In the software, all possible cross-correlations are enumerated and the corresponding histograms allocated before beginning the calculation. This requires some computational overhead at instantiation, and if an application requires the calculation of large numbers of histograms it may be more efficient to modify the code to output and reset upon receiving particular signals.

The implementation of these methods can be found in \texttt{histogram\_gn.c} and \texttt{combinations.c}.

\subsubsection{A histogram is function of $n$-dimensional vectors mapping mapping onto integers}
Once the particular cross-correlation has been identified, the task falls to that of incrementing a counter corresponding to the appropriate histogram bin. The purpose of a histogram is to divide some phase space into well-defined smaller blocks, and then to count the number of events which fall into those blocks. We will restrict ourselves to rectangular volumes, so the task of identifying the correct volume of phase space can be reduced to determining the correct index along all axes, and thus each indexing step is identical to that for a single-dimensional histogram. 

In one dimension, we first define a range of values into which the events can fall, for example $\timewindow=\left[\timewindow\upminus,\timewindow\upplus\right)\subset\integers$. Next, we define some number $N$ of ranges $\resolution$ whose members collectively span $\timewindow$. These $\resolution$ can be enumerated $0,1,\ldots N-1$, so assign these indices in order to the sorted $\resolution$. Now this one dimension is represented by an $N$-dimensional vector whose dimensions represent some $\resolution$. The task then turns to mapping any value $z\in\integers$ to the appropriate $\resolution$.

For linearly-spaced sub-ranges, this identification can be achieved in O(1) time by:
\begin{equation}
\Index(x) = \timewindow\upminus + x\parens{\frac{\timewindow\upplus-\timewindow\upminus}{N}}
\end{equation}
However, in the code this is not used, and instead a more general binary search algorithm (costing O($\log(N)$)) is implemented to permit arbitrary spacings of these ranges. The binary algorithm determines the placement of an element into a sorted list by iterative division of the list into upper and lower halves, until the appropriate element is define. This is implemented as:
\begin{lstlisting}
def binary_search(value, bins):
    upper = len(bins)
    lower = 0
    
    if value < bins[0] or values > bins[-1]:
        # Not in the bounds
        return(False)
     
    while True:
        middle = (upper - lower)/2
        if value in bins[middle]:
            # Just right
            return(middle)
        elif value < bins[middle]:
            # Too high
            upper = middle
        else:
            # Too low
            lower = middle
\end{lstlisting}
Because this must be performed for each dimension of the correlation, the search costs O($n\log{(N)}$) time, compared to the O($n$) for the complete linear search. 

To see how this is implemented in \program{histogram}, see \texttt{histogram\_gn.c} and \texttt{histogram\_t*.c}. 

\subsubsection{Printing the histogram}
Having exhausted the stream of correlation events, the final task is to print all of the histogram bins in a readable format. This is done by iterating over the histograms, then iterating over the bins, printing the counts associated with each bin. How this process is performed should be evident from the preceding discussion, and for the details refer to the routine \texttt{print\_gn\_histogram} in \texttt{histogram\_gn.c}.

Note that the output of $\program{histogram}$ is not normalized in any way: only the number of counts is reported for each bin.



\subsubsection{Input}
\paragraph{\gn{1} of T3 data}
The expected input is a stream of T3 photons.

\paragraph{\gn{n\ge 2}}
The expected input is the output of \program{correlate}.

\paragraph{General options}
The main options specified for \program{histogram} define the axes for pulse and time. These are treated equivalently, so we will focus on the time axes.

For \texttt{--time}, the parameters specify the lower and upper bounds of the time axis, as well as the number of bins $n$ to create along that axis. For a linear spacing, the bin width $\Delta t$ is $(t_{\max}-t_{\min})/n$, such that the bins are defined by the ranges
\begin{equation}
\begin{aligned}
&[\time_{\min},\time_{\min}+\Delta\time),\\
&[\time_{\min}+\Delta\time,\time_{\min}+2\Delta\time),\\
&\ldots\\
&[\time_{\min}+(n-1)\Delta\time,\time_{\min}+n\Delta\time)
\end{aligned}
\end{equation}
This linear spacing is the default behavior, but the flag \texttt{--time-scale} can produce two other scales: log, and log-zero. The log scale creates bins with fixed width over the span of $[\log(\time_{\min}),\log(\time_{\max}))$, as if often desired for measurements requiring long and short time correlations. The log-zero scale has identical behavior, except that any zero-time correlations ($\timedelay=0$) are placed into the first bin. Note that the log scale cannot handle zero-time correlations, and neither log not log-zero can handle negative-time correlations. These values will be dropped from the histogram, with an error message indicating this has happened.

\subsubsection{Output}
After the input stream terminates, \program{histogram} outputs the bin definitions and the number of counts associated with that bin. Generically, this format is:
\begin{verbatim}
channel 0, channel 1, bin (1,1) lower, bin (1,1) upper, ...,
    channel 2, ... , 
    counts \n
\end{verbatim}
where the channels are integers, bin edges are floats, and the counts are integers. For every bin in the histogram, one line will be output. For T2 mode, the bin definition has only one dimension (time), so the format is:
\begin{verbatim}
channel 0, channel 1, time 1 lower, time 1 upper, 
           channel 2, time 2 lower, ...,
           counts \n
\end{verbatim}
T3 data have an additional dimension (pulse):
\begin{verbatim}
channel 0, channel 1, pulse 1 lower, pulse 1 upper,
                      time 1 lower, time 1 uppper,
           channel 2, ...
           counts \n
\end{verbatim}
See section~\ref{sec:histogram_examples} for specific examples of output in these formats.


\subsection{Examples of usage}
\label{sec:histogram_examples}
\subsubsection{Time-averaged photoluminescence lifetime from T3 data}
In T3 mode, a correlation order \gn{1} is code for interactive-like behavior. Formally this is a correlation of the laser pulse and the system response, but this language is not often used.
\begin{verbatim}
> picoquant --file-in data.pt3 |  \
  histogram --mode t3 --order 1 --channels 2 \
            --time 0,10,500000
0,0.00,50000.00,0
0,50000.00,100000.00,0
...
1,0.00,50000.00,11267838
1,50000.00,100000.00,14947845
1,100000.00,150000.00,1512803
1,150000.00,200000.00,1152498
1,200000.00,250000.00,1037717
1,250000.00,300000.00,973572
1,300000.00,350000.00,932802
1,350000.00,400000.00,899615
1,400000.00,450000.00,12278
1,450000.00,500000.00,0
\end{verbatim}
Note that \texttt{histogram} will operate on channel 0 as well, even though channel 1 is the only channel with any signal. This costs extra memory and some computational overhead at startup, but ultimately the cost is insignificant compared to the cost of processing the data stream.

\subsubsection{\gn{2} from T2 data}
This data represents an electronic sync source (channel 4) and the detection of the laser itself.
\begin{verbatim}
> picoquant --file-in data.ht2 | \
  correlate --mode t2 --order 2 \
            --channels 5 \
            --max-time-distance 1000 | \
  histogram --mode t2 --order 2 \
            --channels 5 \
            --time 0,10,1000
0,0,0.00,100.00,0
...
0,4,0.00,100.00,43
0,4,100.00,200.00,24
0,4,200.00,300.00,38
0,4,300.00,400.00,43
0,4,400.00,500.00,44
...
4,4,900.00,1000.00,0
\end{verbatim}

\subsubsection{\gn{2} from T3 data}
\begin{verbatim}
> picoquant --file-in data.pt3 | \
  correlate --mode t3 --channels 2 \
            --order 2 \
            --max-pulse-distance 3 \
  histogram --mode t3 --channels 2 \
            --order 2 --pulse 0,3,3 \
            --time -500000,2,500000
0,0,0.00,1.00,-500000.00,0.00,0
...
1,1,0.00,1.00,-500000.00,0.00,0
1,1,0.00,1.00,0.00,500000.00,104640
1,1,1.00,2.00,-500000.00,0.00,123289
1,1,1.00,2.00,0.00,500000.00,124676
1,1,2.00,3.00,-500000.00,0.00,231962
1,1,2.00,3.00,0.00,500000.00,231694
\end{verbatim}
As a rough benchmark, on a computer with a dual-core 3\giga\hertz{} processor running 32-bit Linux, this command required 25\second{} of wall time, for \texttt{data.ht3} containing 32.7 million photon records (18.1\kilo\cps) for a total of 0.8 million correlation events.

\subsubsection{\gn{3} from T2 data}
This T2 data is the same as the laser data from before:
\begin{verbatim}
> picoquant --file-in data.ht2 | \
  correlate --mode t2 --order 2 \
            --channels 5 \
            --max-time-distance 1000000 | \
  histogram --mode t2 --order 2 \
            --channels 5 \
            --time 0,1,1000000
0,0,0.00,1000000.00,0,0.00,1000000.00,31
0,0,0.00,1000000.00,1,0.00,1000000.00,55
0,0,0.00,1000000.00,2,0.00,1000000.00,57
...
4,0,0.00,1000000.00,2,0.00,1000000.00,2710
4,0,0.00,1000000.00,3,0.00,1000000.00,2235
4,0,0.00,1000000.00,4,0.00,1000000.00,0
...
\end{verbatim}
Note how much larger the time window must be to catch significant numbers of higher-order events.

\subsection{Implementation details}
The problem of populating and returning histograms for all possible correlations can be broken into a few distinct steps:
\begin{enumerate}
\item Construct a histogram for every permutation of channels.
\item For each correlation event:
  \begin{enumerate}
  \item Identify the histogram corresponding to the permutation of channels.
  \item Identify the bin associated with the parameters of the correlation.
  \item Increment that bin in that histogram.
  \end{enumerate}
\item For every bin of every histogram, print the bin definition and associated counts
\end{enumerate}
As such, each of these will be discussed separately. The discussion will focus on T2 data, but T3 data are handled identically for twice the order of equivalent T2 data.

\subsubsection{The cross-correlations can be enumerated as base-$\abs{\channels}$ numbers}
Each histogram corresponds to a single cross-correlation, as identified by the tuple $\vec{\channel}\in\channels^{n}$, for $\channels$ the set of all channels and correlation order $n$. As such, if we enumerate the channels as whole numbers ($0, 1, \ldots$), it is evident that each element of the tuple can be treated as the coefficient of an $n$-digit number in base $\abs{\channels}$. Mapping these tuples onto the set of whole numbers can thus be achieved by the following formula:
\begin{equation}
\Index(\vec{\channel}) = \sum_{j=0}^{n-1}{\channel_{j}\abs{\channels}^{n-1-j}}
\end{equation}
where the elements $\channel$ refer implicitly to the index of that channel in our enumeration.
If $\abs{\channels}=2$, this is identical to the expression of an $n$-digit binary number.

In the software, all possible cross-correlations are enumerated and the corresponding histograms allocated before beginning the calculation. This requires some computational overhead at instantiation, and if an application requires the calculation of large numbers of histograms it may be more efficient to modify the code to output and reset upon receiving particular signals.

The implementation of these methods can be found in \texttt{histogram\_gn.c} and \texttt{combinations.c}.

\subsubsection{A histogram is function of $n$-dimensional vectors mapping mapping onto integers}
Once the particular cross-correlation has been identified, the task falls to that of incrementing a counter corresponding to the appropriate histogram bin. The purpose of a histogram is to divide some phase space into well-defined smaller blocks, and then to count the number of events which fall into those blocks. We will restrict ourselves to rectangular volumes, so the task of identifying the correct volume of phase space can be reduced to determining the correct index along all axes, and thus each indexing step is identical to that for a single-dimensional histogram. 

In one dimension, we first define a range of values into which the events can fall, for example $\timewindow=\left[\timewindow\upminus,\timewindow\upplus\right)\subset\integers$. Next, we define some number $N$ of ranges $\resolution$ whose members collectively span $\timewindow$. These $\resolution$ can be enumerated $0,1,\ldots N-1$, so assign these indices in order to the sorted $\resolution$. Now this one dimension is represented by an $N$-dimensional vector whose dimensions represent some $\resolution$. The task then turns to mapping any value $z\in\integers$ to the appropriate $\resolution$.

For linearly-spaced sub-ranges, this identification can be achieved in O(1) time by:
\begin{equation}
\Index(x) = \timewindow\upminus + x\parens{\frac{\timewindow\upplus-\timewindow\upminus}{N}}
\end{equation}
However, in the code this is not used, and instead a more general binary search algorithm (costing O($\log(N)$)) is implemented to permit arbitrary spacings of these ranges. The binary algorithm determines the placement of an element into a sorted list by iterative division of the list into upper and lower halves, until the appropriate element is define. This is implemented as:
\begin{lstlisting}
def binary_search(value, bins):
    upper = len(bins)
    lower = 0
    
    if value < bins[0] or values > bins[-1]:
        # Not in the bounds
        return(False)
     
    while True:
        middle = (upper - lower)/2
        if value in bins[middle]:
            # Just right
            return(middle)
        elif value < bins[middle]:
            # Too high
            upper = middle
        else:
            # Too low
            lower = middle
\end{lstlisting}
Because this must be performed for each dimension of the correlation, the search costs O($n\log{(N)}$) time, compared to the O($n$) for the complete linear search. 

To see how this is implemented in \program{histogram}, see \texttt{histogram\_gn.c} and \texttt{histogram\_t*.c}. 

\subsubsection{Printing the histogram}
Having exhausted the stream of correlation events, the final task is to print all of the histogram bins in a readable format. This is done by iterating over the histograms, then iterating over the bins, printing the counts associated with each bin. How this process is performed should be evident from the preceding discussion, and for the details refer to the routine \texttt{print\_gn\_histogram} in \texttt{histogram\_gn.c}.

Note that the output of $\program{histogram}$ is not normalized in any way: only the number of counts is reported for each bin.



\subsubsection{Input}
\paragraph{\gn{1} of T3 data}
The expected input is a stream of T3 photons.

\paragraph{\gn{n\ge 2}}
The expected input is the output of \program{correlate}.

\paragraph{General options}
The main options specified for \program{histogram} define the axes for pulse and time. These are treated equivalently, so we will focus on the time axes.

For \texttt{--time}, the parameters specify the lower and upper bounds of the time axis, as well as the number of bins $n$ to create along that axis. For a linear spacing, the bin width $\Delta t$ is $(t_{\max}-t_{\min})/n$, such that the bins are defined by the ranges
\begin{equation}
\begin{aligned}
&[\time_{\min},\time_{\min}+\Delta\time),\\
&[\time_{\min}+\Delta\time,\time_{\min}+2\Delta\time),\\
&\ldots\\
&[\time_{\min}+(n-1)\Delta\time,\time_{\min}+n\Delta\time)
\end{aligned}
\end{equation}
This linear spacing is the default behavior, but the flag \texttt{--time-scale} can produce two other scales: log, and log-zero. The log scale creates bins with fixed width over the span of $[\log(\time_{\min}),\log(\time_{\max}))$, as if often desired for measurements requiring long and short time correlations. The log-zero scale has identical behavior, except that any zero-time correlations ($\timedelay=0$) are placed into the first bin. Note that the log scale cannot handle zero-time correlations, and neither log not log-zero can handle negative-time correlations. These values will be dropped from the histogram, with an error message indicating this has happened.

\subsubsection{Output}
After the input stream terminates, \program{histogram} outputs the bin definitions and the number of counts associated with that bin. Generically, this format is:
\begin{verbatim}
channel 0, channel 1, bin (1,1) lower, bin (1,1) upper, ...,
    channel 2, ... , 
    counts \n
\end{verbatim}
where the channels are integers, bin edges are floats, and the counts are integers. For every bin in the histogram, one line will be output. For T2 mode, the bin definition has only one dimension (time), so the format is:
\begin{verbatim}
channel 0, channel 1, time 1 lower, time 1 upper, 
           channel 2, time 2 lower, ...,
           counts \n
\end{verbatim}
T3 data have an additional dimension (pulse):
\begin{verbatim}
channel 0, channel 1, pulse 1 lower, pulse 1 upper,
                      time 1 lower, time 1 uppper,
           channel 2, ...
           counts \n
\end{verbatim}
See section~\ref{sec:histogram_examples} for specific examples of output in these formats.


\subsection{Examples of usage}
\label{sec:histogram_examples}
\subsubsection{Time-averaged photoluminescence lifetime from T3 data}
In T3 mode, a correlation order \gn{1} is code for interactive-like behavior. Formally this is a correlation of the laser pulse and the system response, but this language is not often used.
\begin{verbatim}
> picoquant --file-in data.pt3 |  \
  histogram --mode t3 --order 1 --channels 2 \
            --time 0,10,500000
0,0.00,50000.00,0
0,50000.00,100000.00,0
...
1,0.00,50000.00,11267838
1,50000.00,100000.00,14947845
1,100000.00,150000.00,1512803
1,150000.00,200000.00,1152498
1,200000.00,250000.00,1037717
1,250000.00,300000.00,973572
1,300000.00,350000.00,932802
1,350000.00,400000.00,899615
1,400000.00,450000.00,12278
1,450000.00,500000.00,0
\end{verbatim}
Note that \texttt{histogram} will operate on channel 0 as well, even though channel 1 is the only channel with any signal. This costs extra memory and some computational overhead at startup, but ultimately the cost is insignificant compared to the cost of processing the data stream.

\subsubsection{\gn{2} from T2 data}
This data represents an electronic sync source (channel 4) and the detection of the laser itself.
\begin{verbatim}
> picoquant --file-in data.ht2 | \
  correlate --mode t2 --order 2 \
            --channels 5 \
            --max-time-distance 1000 | \
  histogram --mode t2 --order 2 \
            --channels 5 \
            --time 0,10,1000
0,0,0.00,100.00,0
...
0,4,0.00,100.00,43
0,4,100.00,200.00,24
0,4,200.00,300.00,38
0,4,300.00,400.00,43
0,4,400.00,500.00,44
...
4,4,900.00,1000.00,0
\end{verbatim}

\subsubsection{\gn{2} from T3 data}
\begin{verbatim}
> picoquant --file-in data.pt3 | \
  correlate --mode t3 --channels 2 \
            --order 2 \
            --max-pulse-distance 3 \
  histogram --mode t3 --channels 2 \
            --order 2 --pulse 0,3,3 \
            --time -500000,2,500000
0,0,0.00,1.00,-500000.00,0.00,0
...
1,1,0.00,1.00,-500000.00,0.00,0
1,1,0.00,1.00,0.00,500000.00,104640
1,1,1.00,2.00,-500000.00,0.00,123289
1,1,1.00,2.00,0.00,500000.00,124676
1,1,2.00,3.00,-500000.00,0.00,231962
1,1,2.00,3.00,0.00,500000.00,231694
\end{verbatim}
As a rough benchmark, on a computer with a dual-core 3\giga\hertz{} processor running 32-bit Linux, this command required 25\second{} of wall time, for \texttt{data.ht3} containing 32.7 million photon records (18.1\kilo\cps) for a total of 0.8 million correlation events.

\subsubsection{\gn{3} from T2 data}
This T2 data is the same as the laser data from before:
\begin{verbatim}
> picoquant --file-in data.ht2 | \
  correlate --mode t2 --order 2 \
            --channels 5 \
            --max-time-distance 1000000 | \
  histogram --mode t2 --order 2 \
            --channels 5 \
            --time 0,1,1000000
0,0,0.00,1000000.00,0,0.00,1000000.00,31
0,0,0.00,1000000.00,1,0.00,1000000.00,55
0,0,0.00,1000000.00,2,0.00,1000000.00,57
...
4,0,0.00,1000000.00,2,0.00,1000000.00,2710
4,0,0.00,1000000.00,3,0.00,1000000.00,2235
4,0,0.00,1000000.00,4,0.00,1000000.00,0
...
\end{verbatim}
Note how much larger the time window must be to catch significant numbers of higher-order events.

\subsection{Implementation details}
The problem of populating and returning histograms for all possible correlations can be broken into a few distinct steps:
\begin{enumerate}
\item Construct a histogram for every permutation of channels.
\item For each correlation event:
  \begin{enumerate}
  \item Identify the histogram corresponding to the permutation of channels.
  \item Identify the bin associated with the parameters of the correlation.
  \item Increment that bin in that histogram.
  \end{enumerate}
\item For every bin of every histogram, print the bin definition and associated counts
\end{enumerate}
As such, each of these will be discussed separately. The discussion will focus on T2 data, but T3 data are handled identically for twice the order of equivalent T2 data.

\subsubsection{The cross-correlations can be enumerated as base-$\abs{\channels}$ numbers}
Each histogram corresponds to a single cross-correlation, as identified by the tuple $\vec{\channel}\in\channels^{n}$, for $\channels$ the set of all channels and correlation order $n$. As such, if we enumerate the channels as whole numbers ($0, 1, \ldots$), it is evident that each element of the tuple can be treated as the coefficient of an $n$-digit number in base $\abs{\channels}$. Mapping these tuples onto the set of whole numbers can thus be achieved by the following formula:
\begin{equation}
\Index(\vec{\channel}) = \sum_{j=0}^{n-1}{\channel_{j}\abs{\channels}^{n-1-j}}
\end{equation}
where the elements $\channel$ refer implicitly to the index of that channel in our enumeration.
If $\abs{\channels}=2$, this is identical to the expression of an $n$-digit binary number.

In the software, all possible cross-correlations are enumerated and the corresponding histograms allocated before beginning the calculation. This requires some computational overhead at instantiation, and if an application requires the calculation of large numbers of histograms it may be more efficient to modify the code to output and reset upon receiving particular signals.

The implementation of these methods can be found in \texttt{histogram\_gn.c} and \texttt{combinations.c}.

\subsubsection{A histogram is function of $n$-dimensional vectors mapping mapping onto integers}
Once the particular cross-correlation has been identified, the task falls to that of incrementing a counter corresponding to the appropriate histogram bin. The purpose of a histogram is to divide some phase space into well-defined smaller blocks, and then to count the number of events which fall into those blocks. We will restrict ourselves to rectangular volumes, so the task of identifying the correct volume of phase space can be reduced to determining the correct index along all axes, and thus each indexing step is identical to that for a single-dimensional histogram. 

In one dimension, we first define a range of values into which the events can fall, for example $\timewindow=\left[\timewindow\upminus,\timewindow\upplus\right)\subset\integers$. Next, we define some number $N$ of ranges $\resolution$ whose members collectively span $\timewindow$. These $\resolution$ can be enumerated $0,1,\ldots N-1$, so assign these indices in order to the sorted $\resolution$. Now this one dimension is represented by an $N$-dimensional vector whose dimensions represent some $\resolution$. The task then turns to mapping any value $z\in\integers$ to the appropriate $\resolution$.

For linearly-spaced sub-ranges, this identification can be achieved in O(1) time by:
\begin{equation}
\Index(x) = \timewindow\upminus + x\parens{\frac{\timewindow\upplus-\timewindow\upminus}{N}}
\end{equation}
However, in the code this is not used, and instead a more general binary search algorithm (costing O($\log(N)$)) is implemented to permit arbitrary spacings of these ranges. The binary algorithm determines the placement of an element into a sorted list by iterative division of the list into upper and lower halves, until the appropriate element is define. This is implemented as:
\begin{lstlisting}
def binary_search(value, bins):
    upper = len(bins)
    lower = 0
    
    if value < bins[0] or values > bins[-1]:
        # Not in the bounds
        return(False)
     
    while True:
        middle = (upper - lower)/2
        if value in bins[middle]:
            # Just right
            return(middle)
        elif value < bins[middle]:
            # Too high
            upper = middle
        else:
            # Too low
            lower = middle
\end{lstlisting}
Because this must be performed for each dimension of the correlation, the search costs O($n\log{(N)}$) time, compared to the O($n$) for the complete linear search. 

To see how this is implemented in \program{histogram}, see \texttt{histogram\_gn.c} and \texttt{histogram\_t*.c}. 

\subsubsection{Printing the histogram}
Having exhausted the stream of correlation events, the final task is to print all of the histogram bins in a readable format. This is done by iterating over the histograms, then iterating over the bins, printing the counts associated with each bin. How this process is performed should be evident from the preceding discussion, and for the details refer to the routine \texttt{print\_gn\_histogram} in \texttt{histogram\_gn.c}.

Note that the output of $\program{histogram}$ is not normalized in any way: only the number of counts is reported for each bin.


\section{Applications}
Now that we have laid the framework for the efficient handling of photon-arrival data, we can apply the results to interesting scientific problems.

\subsection{Fluorescence blinking}
In many fluorescent molecules, the emission is not steady at the single-molecule level, and it is useful to characterize this intermittency. One standard method for accomplishing this is to estimate the rate of photon arrival at fixed intervals in time, then determine the rate at which this rate fluctuates across some threshold. The procedure is as follows:
\begin{enumerate}
\item Define a series of fixed-width time bins for counting photons.
\item Determine how many photons arrive in each bin, and normalize the counts to determine an average rate of photon arrivals in that bin.
\item Given a threshold, map the count rate in each bin to 0 (below threshold) or 1 (above).
\item Determine the duration of each on (1) or off (0) period, and compute the histogram of these durations.
\end{enumerate}

For an example implementation. see \texttt{scripts/blinking.py}.

\subsection{Time-dependent photoluminescence lifetime}
The PL lifetime can be recovered from T3 data by calculating the order 1 histogram:
\begin{verbatim}
> picoquant --file-in data.pt3 | \
  histogram --mode t3 \
            --order 1 \
            --channels 2 \
            --time 0,10000,100000000
\end{verbatim}
In some situations it is desirable to calculate the PL lifetime for some interval of time and compare that with the lifetime at other intervals. This can be achieved by dividing the photon stream and histogramming the result accordingly. See \texttt{scripts/time\_dependent\_pl.py} for more details.

\subsection{Photon correlation spectroscopy}
By calculating the cross-correlation of two detection channels, we can study important properties of a signal, such as the number of fluorophores represented. This ultimately means we need to calculate \gn{2} for the signal, with the following steps:
\begin{enumerate}
\item Generate the stream of correlation events from the stream of photons.
\item Bin the events to form the cross-correlations.
\item Use the intensities at each channel to normalize the cross-correlation.
\end{enumerate}
See \GN{} for details of the implementation.

\subsubsection{Bunching and antibunching}
For t2 data, calculating \gn{2} with linear bin spacing should be sufficient. For t3 data, choosing a pulse bin width of 1 will enable direct comparison of the response to a single or neighboring pulses, as used in multi-exciton emission studies.

\subsubsection{Fluorescence correlation spectroscopy}
The correlator is reasonably fast and can handle the long time delays required for FCS measurements. One slight modification is that we will typically want a logarithmic time axis, which can be achieved by passing the flag \texttt{--time-scale log} to \GN{} or \program{histogram}. Additionally, it is often useful to perform this measurement with the full autocorrelation, since the time delays are well beyond the dead time of the hardware, so use the full autocorrelation returned by \GN.
\end{document}
