\section{Intensity\_to\_t2}
\subsection{Purpose}
This program converts a stream of time-tagged intensity to T2 photons, permitting a stream of integer counts to be treated as though it were composed of individual photons.

\subsection{Command-line syntax}
%\section{Intensity\_to\_t2}
\subsection{Purpose}
This program converts a stream of time-tagged intensity to T2 photons, permitting a stream of integer counts to be treated as though it were composed of individual photons.

\subsection{Command-line syntax}
\begin{verbatim}
Usage: intensity_to_t2 [options]

Version 1.6

             -V, --verbose: Print debug-level information.
                -h, --help: Prints this usage message.
             -v, --version: Print version information.
             -i, --file-in: Input filename. By default, this is stdin.
            -o, --file-out: Output filename. By default, this is stdout.
            -c, --channels: The number of channels in the signal.

This program accepts intensity data of the form:
      time (long long), counts (unsigned int)
and outputs an equivalent t2 photon stream, tagged on some number
of channels. By default, all photons are sent to channel 0, but 
if a finite number of channels is specified the photons are 
randomly distributed among the possible channels.
\end{verbatim}

\subsubsection{Input}
The input is a stream of time-intensity pairs:
\begin{verbatim}
    time (long long), counts (unsigned int)
\end{verbatim}

\subsubsection{Output}
The output is a stream of T2 photons generated from the given intensity at each time. By default, all photons are assigned to channel 0, but specifying some number of channels spreads the photons uniformly over those channels.

\subsection{Examples of usage}
\subsubsection{Photons on a single channel}
\begin{verbatim}
> cat counts
1,2
2,0
3,1
...
> intensity_to_t2 --file-in counts
0,1
0,1
0,3
...
\end{verbatim}

\subsubsection{Photons on several channels}
\begin{verbatim}
> intensity_to_t2 --file-in counts --channels 4
1,1
2,1
2,3
...
\end{verbatim}

\subsubsection{Input}
The input is a stream of time-intensity pairs:
\begin{verbatim}
    time (long long), counts (unsigned int)
\end{verbatim}

\subsubsection{Output}
The output is a stream of T2 photons generated from the given intensity at each time. By default, all photons are assigned to channel 0, but specifying some number of channels spreads the photons uniformly over those channels.

\subsection{Examples of usage}
\subsubsection{Photons on a single channel}
\begin{verbatim}
> cat counts
1,2
2,0
3,1
...
> intensity_to_t2 --file-in counts
0,1
0,1
0,3
...
\end{verbatim}

\subsubsection{Photons on several channels}
\begin{verbatim}
> intensity_to_t2 --file-in counts --channels 4
1,1
2,1
2,3
...
\end{verbatim}